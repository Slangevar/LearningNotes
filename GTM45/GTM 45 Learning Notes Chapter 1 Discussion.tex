% Options for packages loaded elsewhere
\PassOptionsToPackage{unicode}{hyperref}
\PassOptionsToPackage{hyphens}{url}
%
\documentclass[
]{article}
\usepackage{amsmath,amssymb}
\usepackage{lmodern}
\usepackage{iftex}
\ifPDFTeX
  \usepackage[T1]{fontenc}
  \usepackage[utf8]{inputenc}
  \usepackage{textcomp} % provide euro and other symbols
\else % if luatex or xetex
  \usepackage{unicode-math}
  \defaultfontfeatures{Scale=MatchLowercase}
  \defaultfontfeatures[\rmfamily]{Ligatures=TeX,Scale=1}
\fi
% Use upquote if available, for straight quotes in verbatim environments
\IfFileExists{upquote.sty}{\usepackage{upquote}}{}
\IfFileExists{microtype.sty}{% use microtype if available
  \usepackage[]{microtype}
  \UseMicrotypeSet[protrusion]{basicmath} % disable protrusion for tt fonts
}{}
\makeatletter
\@ifundefined{KOMAClassName}{% if non-KOMA class
  \IfFileExists{parskip.sty}{%
    \usepackage{parskip}
  }{% else
    \setlength{\parindent}{0pt}
    \setlength{\parskip}{6pt plus 2pt minus 1pt}}
}{% if KOMA class
  \KOMAoptions{parskip=half}}
\makeatother
\usepackage{xcolor}
\setlength{\emergencystretch}{3em} % prevent overfull lines
\providecommand{\tightlist}{%
  \setlength{\itemsep}{0pt}\setlength{\parskip}{0pt}}
\setcounter{secnumdepth}{-\maxdimen} % remove section numbering
\ifLuaTeX
  \usepackage{selnolig}  % disable illegal ligatures
\fi
\IfFileExists{bookmark.sty}{\usepackage{bookmark}}{\usepackage{hyperref}}
\IfFileExists{xurl.sty}{\usepackage{xurl}}{} % add URL line breaks if available
\urlstyle{same} % disable monospaced font for URLs
\hypersetup{
  hidelinks,
  pdfcreator={LaTeX via pandoc}}

\author{}
\date{}

\begin{document}

\hypertarget{gtm-45-probability-theory-i}{%
  \section{GTM 45 Probability Theory
    I}\label{gtm-45-probability-theory-i}}

\hypertarget{chapter-1-exercise}{%
  \subsection{Chapter 1 Exercise}\label{chapter-1-exercise}}

The measurable sets belong to a fixed \(\sigma\)-field on which the set
functions and limits of their sequences are defined. Unless otherwise
stated and with or without affixes, \(A, B, ...\) denote sets, \(\mu\)
denotes a measure, \(\varphi\) denotes a signed measure. Notice that a
signed measure is \(\sigma\)-additive.

A nonnegative additive set function is called a content or a measure
according as it is finitely additive or \(\sigma\)-additive. Signed
measures are differences of two measures of which one at least is
finite.

\begin{center}\rule{0.5\linewidth}{0.5pt}\end{center}

\hypertarget{exercise-1-discussion}{%
  \subsubsection{Exercise 1 (Discussion)}\label{exercise-1-discussion}}

If \(\varphi\) is \(\sigma\)-finite, then there are only countably many
disjoint sets for which \(\varphi \neq 0\) in every class.

\emph{Note: A set function \(\varphi\) is defined on a nonempty class
  \(\mathcal{C}\) of sets in a space \(\Omega\) by assigning to every set
  \(A \in \mathcal{C}\) a single number \(\varphi(A)\), finite or
  infinite, the value of \(\varphi\) at \(A\). If all values of
  \(\varphi\) are finite, \(\varphi\) is said to be finite, and we write
  \(|\varphi|<\infty\). If every set in \(\mathcal{C}\) is a countable
  union of sets in \(\mathcal{C}\) at which \(\varphi\) is finite,
  \(\varphi\) is said to be \(\sigma \)-finite.}

\textbf{Solution}

We only considering a \(\sigma\)-field \(\mathfrak{a}\) over the space
\(\Omega\). The proof is as follows.

According to the given condition, we could write

\[\Omega = \bigcup_{i=1}^\infty A_i\]

where \(0 \le \left|\varphi(A_i)\right| < \infty\). Based on the
definition of signed measure, one of \(\varphi^+\) and \(\varphi^-\)
should be finite. Therefore, we know that
\(0 \le \varphi^+(A_i) < \infty\) and \(0 \le \varphi^-(A_i) < \infty\).

Now, assume that there are uncountably many disjoint sets for which
\(\varphi\ne 0\). Then there must be uncountably many disjoints set for
which \(\varphi^+ \ne 0\) or \(\varphi^- \ne 0\) (or both). Without loss
of generality, assume \(\varphi^+ \ne 0\) (the following will be the
same for the case where \(\varphi^- \ne 0\)). Denote these uncountably
many sets as \(B_u, u \in \mathcal U\) where \(\mathcal U\) is some
index set. We will show that it is impossible and so the proposition for
this exercise is correct.

For an arbitrary \(i\), denote \(B_{u,i} = B_u \cap A_i\). Then clearly
\(A_i \supset \cup_{\mathcal U} B_{u, i}\). Without loss of generality,
assume \(\varphi^+(A_i) = M < \infty\). For any \(N \in Z_+\), we argue
that

\[|C_N| \le NM, \ \text{where } C_N = \left\{B_{u, i}: \varphi^+(B_{u,i}) > \frac{1}{N}\right\}\]

Otherwise, take an arbitrary countable subset \(\widetilde C_N\) of
\(C_N\), by the additivity,

\[\varphi^+\left(\bigcup_{\widetilde B_{u, i}\in \widetilde C_N} \widetilde B_{u, i}\right) > \frac{1}{N}NM = M = \varphi^+(A_i)\]

This is contradictory since

\[\bigcup_{\widetilde B_{u, i}\in \widetilde C_N} \widetilde B_{u, i} \subset  \cup_{\mathcal U} B_{u, i} \subset A_i \implies \varphi^+\left(\bigcup_{\widetilde B_{u, i}\in \widetilde C_N} \widetilde B_{u, i}\right) \le \varphi^+(A_i).\]

Therefore, since countably union of countable sets are countable, and
that \(\varphi^+(B_{u}) > 0\) if there exists an \(i\) such that
\(\varphi^+(B_{u, i}) > 0\),

\[\left|\left\{B_u: \varphi^+(B_u) \right\}\right| = \left|\bigcup_{i=1}^\infty\left\{B_{u, i}: \varphi^+(B_{u,i}) > 0\right\}\right| = \left|\bigcup_{i=1}^\infty\bigcup_{N=1}^\infty C_N \right| = \aleph_0,\]

which means there has to be only countably many
\(B_u\)\textquotesingle s such that \(\varphi^+(B_u) >0\). \(\square\)

\begin{center}\rule{0.5\linewidth}{0.5pt}\end{center}

\hypertarget{exercise-2}{%
  \subsubsection{Exercise 2}\label{exercise-2}}

For every \(A\) there exists a \(B\subset A\) such that
\(\bar{\varphi}(A)\le 2|\varphi(B)|\).

\emph{Note 1: Let \(\varphi\) be an additive function on a field
  \(\mathcal{c}\) and define \(\varphi^{+}\)and \(\varphi^{-}\)on
  \(\mathcal c\) by}

\[\varphi^{+}(A)=\sup _{B \subset A} \varphi(B), \quad \varphi^{-}(A)=-\inf _{B \subset A} \varphi(B), \quad A, B \in \mathcal{C} .\]

\emph{The set functions \(\varphi^{+}, \varphi^{-}\)and
  \(\bar{\varphi}=\varphi^{+}+\varphi^{-}\)are called the upper, lower,
  and total variation of \(\varphi\) on \(\mathcal{C}\), respectively.
  Since \(\varphi(\emptyset)=0\), these variations are nonnegative.}

\emph{Note 2: \textbf{Jordan-Hahn decomposition theorem}}

\emph{If \(\varphi\) on a \(\mathfrak{a}\)-feld \(\mathfrak{a}\) is
  \(\sigma\)-additive, then there exists a set \(D\) such that, for every
  \(A \in \mathfrak{a}\),}

\[-\varphi^{-}(A)=\varphi(A D), \quad \varphi^{+}(A)=\varphi\left(A D^c\right) .\]

\emph{\(\varphi^{+}\)and \(\varphi^{-}\)are measures and
  \(\varphi=\varphi^{+}-\varphi^{-}\)is a signed measure.}

\textbf{Solution}

Notice that \(\varphi(A) = \varphi(AD) + \varphi(AD^c)\) by additivity
so that it will always be true that \(\varphi = \varphi^+ - \varphi^-\).

Based on Jordan-Hahn decomposition theorem, we could always find a
\(D\in \mathfrak{a}\) such that for any \(T\in \mathfrak{a}\),
\(-\varphi^-(T) = \varphi(TD)\) and \(\varphi^+(T) = \varphi(TD^c)\).

Now if \(\varphi^-(A) \le \varphi^+(A)\), we let \(B = AD^c\). Then we
know that \(\varphi^+(B) = \varphi^+(A)\) since \(AD^c\cap D^c = AD^c\).
On the other hand, we have
\(\varphi^-(B) = -\varphi(AD^c\cap D) = -\varphi(\empty) = 0\).
Therefore,

\[2|\varphi(B)| = 2|\psi^+(B) - \psi^-(B)| = 2\psi^+(B) = 2\psi^+(A)\]

On the other hand,

\[\bar \varphi(A) = \psi^+(A) + \psi^-(A) \le 2 \psi^+(A)\]

So \(B\) satisfies the result.

Things are completely an analogy when \(\varphi^-(A) > \varphi^+(A)\),
just let \(B = AD\). \(\square\)

\begin{center}\rule{0.5\linewidth}{0.5pt}\end{center}

\hypertarget{exercise-3}{%
  \subsubsection{\texorpdfstring{Exercise 3
    }{Exercise 3 }}\label{exercise-3}}

If \(\varphi_1 \le \varphi_2\), then
\(\varphi_1^{+} \le \varphi_2^{+}, \varphi_1^{-} \ge \varphi_2^{-}\). If
\(\varphi=\varphi_1 \pm \varphi_2\), then
\(\varphi^{\pm} \le \varphi_1 ^\pm+\varphi_2 ^\pm\).

\textbf{Solution}

The first two inequalities are very obvious since
\(\sup_{B\subset A} \varphi_1 \le \sup_{B\subset A} \varphi_2\) and
\(\inf_{B\subset A} \varphi_1 \le \inf_{B\subset A} \varphi_2 \) (then
taking the negation will change the direction of the inequality).

For the third inequality, first consider the case when
\(\varphi = \varphi_1 + \varphi_2\), then we know that

\[\sup_{B\subset A} \varphi_1(B) + \varphi_2(B) \le  \sup_{B_1\subset A} \varphi(B_1) + \sup_{B_2\subset A} \varphi(B_2)\]

and

\[-(\inf_{B\subset A} \varphi_1(B) + \varphi_2(B)) \ge  -\inf_{B_1\subset A} \varphi(B_1) - \inf_{B_2\subset A} \varphi(B_2)\]

Therefore, the inequality to prove is correct. For the case when
\(\varphi = \varphi_1 - \varphi_2\), we also have

\[\sup_{B\subset A} \varphi_1(B) - \varphi_2(B) \le  \sup_{B_1\subset A} \varphi(B_1) - \inf_{B_2\subset A} \varphi(B_2)\]

and

\[\inf_{B\subset A} \varphi_1(B) - \varphi_2(B) \ge  \inf_{B_1\subset A} \varphi(B_1) - \sup_{B_2\subset A} \varphi(B_2)\]

\(\square\)

\begin{center}\rule{0.5\linewidth}{0.5pt}\end{center}

\hypertarget{exercise-4}{%
  \subsubsection{Exercise 4}\label{exercise-4}}

\textbf{Minimality of the Jordan-Hahn decomposition.} If
\(\varphi=\mu^{+}-\mu^{-}\), then \(\varphi^{\pm} \le \mu^{\pm}\)

\textbf{Solution}

It follows the same logic as that in the previous exercise! \(\square\)

\begin{center}\rule{0.5\linewidth}{0.5pt}\end{center}

We say that \(A\) is a \(\varphi\)-null set, if \(\varphi=0\) on
\(\left\{A A^{\prime}, A^{\prime} \in \mathfrak{a}\right\}\). We say
that \(A\) and \(B\) are \(\varphi\)-equivalent, if they coincide up to
a \(\varphi\)-null set. We say that a nonempty set is a
\(\varphi\)-atom, if every measurable subset of \(A\) is
\(\varphi\)-equivalent either to \(\emptyset\) or to \(A\).

\hypertarget{exercise-5}{%
  \subsubsection{Exercise 5}\label{exercise-5}}

The \(\varphi\)-null sets form a \(\sigma\)-ring; the \(\varphi\)-null
sets of \(\varphi\) and of \(\bar{\varphi}\) are the same. The
\(\varphi\)-equivalence is an equivalence relation (reflexive,
transitive, and symmetric), and \(\mathfrak{a}\) splits into
\(\varphi\)-equivalence classes.

\emph{Note: Let \(\mathcal{R}\) be a nonempty collection of sets. Then
  \(\mathcal{R}\) is a \(\sigma\)-ring if:}

\begin{enumerate}
  \def\labelenumi{\arabic{enumi}.}
  \item
        \emph{Closed under countable unions:
          \(\bigcup_{n=1}^{\infty} A_n \in \mathcal{R}\) if
          \(A_n \in \mathcal{R}\) for all \(n \in \mathbb{N}\)}
  \item
        \emph{Closed under relative complementation:
          \(A \backslash B \in \mathcal{R}\) if \(A, B \in \mathcal{R}\)}
\end{enumerate}

\textbf{Solution}

For \(\sigma\)-ring, just verify the two properties. First of all, we
let \(A_n\)\textquotesingle s\textquotesingle{} be \(\varphi\)-null
sets, denote the collection of them as \(\mathcal R\). Then for any
\(A^\prime \in \mathfrak{a}\), we can construct a sequence
\(B_n = A_n - \cup_{i=1}^{n-1} A_i\). Since
\(\cup_{i=1}^{n-1} A_i\in\mathfrak{a}\), we have to have for any
\(A^\prime \in \mathfrak{a}\),

\begin{aligned}
  \varphi(B_nA^\prime)
   & = \varphi(A_nA^\prime) - \varphi\left(\left(\cup_{i=1}^{n-1} A_i \cap A\right) A^\prime\right)      \\
   & =  \varphi(A_nA^\prime) - \varphi\left(A\cap\left(\cup_{i=1}^{n-1} A_i \cap  A^\prime\right)\right) \\
   & = 0 - 0 = 0
\end{aligned}

Therefore

\begin{aligned}
  \varphi\left(\left(\bigcup_{n=1}^\infty A_n\right)A^\prime\right)
   & = \varphi\left(\bigcup_{n=1}^\infty\left(A_nA^\prime\right)\right)   \\
   & =   \varphi\left(\bigcup_{n=1}^\infty\left(B_nA^\prime\right)\right) \\
   & = \sum_{n=1}^\infty\varphi(B_nA^\prime) = 0
\end{aligned}

So we prove that the first property holds. For the second propery, if
\(A_1, A_2 \in \mathcal R\), then for any \(A^\prime\in \mathfrak{a}\),

\begin{aligned}
  \varphi((A_1 - A_2) A^\prime) & = \varphi(A_1 A^\prime) - \varphi(A_1\cap(A_2\cap A^\prime)) = 0 - 0 = 0
\end{aligned}

which means it also holds. Therefore, \(\varphi\)-null sets for a
\(\sigma\)-ring.

To show that the \(\varphi\)-null sets of \(\varphi\) and of
\(\bar{\varphi}\) are the same, we again use the Jordan-Hahn
decomposition theorem. For any \(\bar\varphi\)-null set \(A\), we have
there exists a \(D\in\mathfrak{a}\), for any
\(A^\prime \in \mathfrak{a}\)

\[\psi(AA^\prime D^c) + \psi(AA^\prime D) = 0 \Rightarrow \psi(AA^\prime) = 0\]

Therefore, we take \(A^\prime = A^\prime D\) and
\(A^\prime = A^\prime D^c\) to get
\(\varphi(AA^\prime D^c) = \varphi(AA^\prime D) = 0\) which means \(A\)
is also \( \varphi\)-null.

On the other hand, if \(A\) is \(\varphi\)-null, then

\[\psi(AA^\prime D^c) - \psi(AA^\prime D) = 0 \Rightarrow\psi(AA^\prime D^c ) = \psi(AA^\prime D)\]

Take \(A^\prime = A^\prime D\) we will have that
\(\varphi(\empty) = \varphi(AA^\prime D) = 0\) and take
\(A^\prime = A^\prime D^c\) we will have
\(\varphi(AA^\prime D^c) = \varphi(\empty) = 0\). Therefore, we must
have \(A\) is also \(\bar\varphi\)-null.

The reflexitivity and the symmetry of \(\varphi\)-equivalence are
trivial. We only need to show for the transitivity of it. Suppose \(A\)
and \(B\) are \(\varphi\)-equivalent, and \(B\) and \(C\) are
\(\varphi\)-equivalent.

\begin{aligned}
  A - C & = (A - B) - (C-B) \cup (A\cap (B - C))
\end{aligned}

Since \(\varphi\)-null sets form a \(\sigma\)-ring, we know that
\((A - B) - (C -B)\) is also \(\sigma\)-null since the two parenthesized
terms are both \(\sigma\)-null. On the other hand, Then for any
\(A^\prime \in \mathfrak{a}\)

\[\varphi(A\cap(B-C)A^\prime) = \varphi((B-C)\cap(AA^\prime)) = 0\]

Therefore, by the property of a \(\sigma\)-ring, we can conclude that
\(A-C\) is also \(\sigma\)-null, which leads to transitivity.

For the last statement, for every set in \(\mathfrak{a}\), we could
always categorize it into some class where sets in that class are
\(\varphi\)-equivalent. It itself could form such a class without any
other members, otherwise if it is \(\varphi\)-equivalent to some set in
other \(\varphi\)-equivalence class, then by transitivity, it is
\(\varphi\)-equivalence to all the other sets in this class, then the
two classes could be merged. \(\square\)

\begin{center}\rule{0.5\linewidth}{0.5pt}\end{center}

\hypertarget{exercise-6}{%
  \subsubsection{Exercise 6}\label{exercise-6}}

Every \(\varphi\)-null set and every measurable set consisting of one
point is a \(\varphi\)-atom; \(\bar{\varphi}(A)=|\varphi(A)|\) for every
\(\varphi\)-atom \(A\). Atoms of \(\varphi\) and \(\bar{\varphi}\) are
the same; atoms of \(\varphi\) are atoms of \(\varphi^{+}\)and
\(\varphi^{-}\), but the converse is not necessarily true.

If \(A\) is a \(\varphi\)-atom, then \(\varphi=0\) or \(\varphi(A)\) on
\(A \cap \mathfrak{a} ;\) if \(\varphi\) is finite, then the converse is
true. What if \(\varphi\) is \(\sigma\)-finite? What about
\(\varphi=\infty\) except for \(\emptyset\) ?

\textbf{Solution}

For measurable set consisting of one point, the conclusion is trivial
-\/- its measurable subsets could only be \(\empty\) and itself! For a
\(\varphi\)-null set \(A\), consider its measurable subset \(B\), then
we have for any \(A^\prime\) in \(\mathfrak{a}\),

\[\varphi((A - B)A^\prime) = \varphi(AA^\prime) - \varphi((A\cap B)A^\prime) = \varphi(AA^\prime) - \varphi(A\cap(B\cap A^\prime)) = 0-0 = 0\]

Therefore, its subset is always \(\varphi\)-equivalent with \(A\) itself
and so \(A\) is a \(\varphi\)-atom.

For the second statement, based on Jordan-Hahn decomposition theorem, we
want to show that

\[\psi(AD) + \psi(AD^c) = |\psi(AD) - \psi(AD^c)|\]

Since \(A\) is a \(\varphi\)-atom, we have that either \(A - AD^c = AD\)
is \(\varphi\)-null or \(AD^c\) is \(\varphi\)-null. For the former
case, we have \(\forall A^\prime \in \mathfrak{a}\)

\[\psi(A^\prime AD) = 0 \Rightarrow \psi(AD) = 0.\]

For the latter case, we have \(\forall A^\prime\in \mathfrak{a}\),

\[\psi(A^\prime AD^c) = 0 \Rightarrow \psi(AD^c) = 0\]

For both cases, it is easy to check that the aforementioned equality
(17) holds.

For the third statement, from the solution for exercise 5, we know that
a \(\varphi\)-null set is also \(\varphi^+\) and \(\varphi^-\)-null.
Therefore, atoms of \(\varphi\) will be the atoms of \(\varphi^+\) and
\(\varphi^-\). But there is no such relationship conversely.

If \(A\) is a \(\varphi\)-atom, then of course \(\varphi=0\) when \(A\)
is \(\varphi\)-null. In general, for any \(B\in A\cap \mathfrak{a}\), we
have \(B\) being \(\varphi\)-equivalent to \(\emptyset\) or \(A\). For
the former case, we pick \(A^\prime = B\), then

\[\varphi((B-\emptyset)B) = 0 \Rightarrow \varphi(B) = 0.\]

For the latter case, we consider \(B^c = A - B\) and pick
\(A^\prime = A - B\), then

\[\varphi(A - B) = 0\]

Since \(\varphi(A) = \varphi(A-B) + \varphi(B)\), we have
\(\varphi(B) = \varphi(A)\).

If \(\varphi\) is finite, we need to show that \(A\) is a
\(\varphi\)-atom given \(\varphi=0\) or \(\varphi(A)\) on
\(A\cap \mathfrak{a}\). For any measurable subset \(B \in A\), we know
that \(\varphi(B) = 0 \text{ or } \varphi(A)\).

For the former case, we know that for every
\(A^\prime \in \mathfrak{a}\),
\((B - \empty)A^\prime  = BA^\prime \subset A\) so that
\(\varphi(BA^\prime) = 0 \text{ or } \varphi(A)\). On the other hand, we
know that \(\varphi(BA^\prime) = \varphi(B) - \varphi(B(A^\prime)^c)\).
What is more, following the same argument,
\(\varphi(B(A^\prime)^c) = 0 \text{ or } \varphi(A)\). Therefore, we
must have \(\varphi(BA^\prime) = 0\) so that \(B\) is
\(\varphi\)-equivalent to \(\empty\).

For the latter case, \(\varphi(B) = \varphi(A)\). Consider the set
\(A - B\). \(\forall A^\prime \in \mathfrak{a}\),
\((A - B)A^\prime \subset A\) , we want to show that
\(\varphi((A - B)A^\prime) = 0 \). Suppose not, then we must have
\(\varphi((A-B)A^\prime) = \varphi(A)\). On the other hand, we know that
\(\varphi(A - B) = \varphi(A) - \varphi(B) = 0\) and
\(\varphi((A-B)(A^\prime)^c) = 0 \text{ or } \varphi(A)\). Since

\[\varphi((A-B)A^\prime) + \varphi((A-B)(A^\prime)^c) = \varphi(A-B)\]

The supposed case will lead to contradictory result. In consequence, we
must have \(\varphi((A-B)A^\prime) = 0\). Which means that \(B\) is
\(\varphi\)-equivalent to \(A\). Hence, \(A\) is a \(\varphi\)-atom.

If \(\varphi\) is \(\sigma\)-finite, then our previous argument for the
case when \(\varphi(B) = 0\) is still valid. When
\(\varphi(B) = \varphi(A)\), it is possible that
\(\varphi(B) = \varphi(A) = \infty\), only for this case
can\textquotesingle t we use the previous arguments. But this is
impossible. Based on the definition of \(\sigma\)-finite, let us
decompose \(A\) into \(\sum_{i=1}^\infty A_i\) where
\(\varphi(A_i) < \infty, \forall i\), which means, \(\varphi(A_i) = 0\).
This means that \(\varphi(A)\) = 0 which is contradictory. Hence, \(A\)
is a \(\varphi\)-atom.

If \(\varphi = \infty\) except for \(\empty\), then the conclusion is
not true since it could only be true when
\(\varphi(BA^\prime) \text{ or } \varphi((A-B)A^\prime)\) is always
\(0\), which means \(BA^\prime\) or \((A-B)A^\prime\) will always be
\(\empty\), which is clearly impossible. \(\square\)

\begin{center}\rule{0.5\linewidth}{0.5pt}\end{center}

\hypertarget{exercise-7-discussion}{%
  \subsubsection{Exercise 7 (Discussion)}\label{exercise-7-discussion}}

If \(\mu\) is finite, then \(\Omega=\sum A_j+A\) where the \(A_j\) or
\(A\) may be absent but, if present, then the \(A_j\) are \(\mu\)-atoms
of positive measure and, for every \(B \subset A\) of positive measure,
\(\mu\) takes every value \(c\) between 0 and \(\mu B\) for measurable
subsets of \(B\). This decomposition of \(\Omega\) is determined up to
\(\mu\)-null sets. Can \(\mu\) be replaced by \(\varphi\) ?

(There is only a countable number of \(\mu\)-equivalence classes of such
\(A_j\) \textquotesingle s. Select representatives \(A_j\) of these
classes and let \(B \subset A=\Omega-\sum A_j\). Select inductively sets
\(C_n \in \mathcal{C}_n\) such that \(\mu C_n>\sup \mu C-\frac{1}{n}\)
for all \(C \in \mathcal{C}_n\), where \(\mathcal{C}_n\) is the class of
all \(C \subset B-\left(C_1 \cup C_2 \ldots \cup C_{n-1}\right)\) for
which \(\mu C \leqq c-\)
\(\mu\left(C_1 \cup C_2 \cup \ldots \cup C_{n-1}\right)\). Then
\(\mu C=c\) for \(\left.C=\bigcup C_{n .}\right)\)

\textbf{Solution}

According to the statement, only \(A\) could be non-atomic.
\(\Omega=\sum A_j+A\) is given so we do not prove that \(A_j\) is
\(\mu\)-atoms. Now suppose we have such \(A_j\)\textquotesingle s, then
clearly there are only coutably many of \(\mu\)-equivalent classes of
them. Otherwise, uncountable sum of positive values will lead to
infinite value of \(\mu(\Omega)\) which is contradictory. Select
representatives \(A_j\) of these classes and let
\(B \subset A=\Omega-\sum A_j\).

We now actually wants to show that \(A\) could be "infinitely divided"
into sets with smaller measures. Based on the hint, we can inductively
select sets. Let \(n = 1\), then can we select a set \(C_1\) from \(B\)
such that \(C_1 \in \mathcal{C}_1\) such that \(\mu C_1>\sup \mu C-1\)
for all \(C \in \mathcal{C}_1\), where \(\mathcal{C}_1\) is the class of
all \(C \subset B\) for which \(\mu C \le c\). We only need to show that
there are sets satisfying \(\mu C \le c\) and the rest is obvious since
we could always select the that \(C_1\) based on the definition of
\(\sup\).

If there is no \(C\subset B\) such that \(\mu C \le c\), then it will
lead to a contradiction since it means that there is also no sets
\(D\subset\) satisfying \(\mu D \ge \mu B - c\). We can take a sequence
\(D_n \to B\), so that \(B-D_n \to \emptyset\) while
\(\mu(B-D_n) \ge \mu B -c > 0\), which is impossible since \(\mu\) is
continuous (\emph{CONTINUITY THEOREM FOR ADDITIVE SET FUNCTIONS}). So
that we can always find \(C\)\textquotesingle s with \(\mu C \le c\).

Similarly, we could find \(C_2\), \(C_3\), \(\dots\) We know that
\(\mu\left(\bigcup_{i=1}^n C_n\right)\) is an increasing sequence.
Moreover, by the construction we know that
\(\mu\left(\bigcup_{i=1}^n C_n\right) > c - \frac{1}{n}\) since
\(\sup \mu C \ge c - \mu\left(C_1 \cup C_2 \cup \ldots \cup C_{n-1}\right)\).
(Actually each \(C_i\) is disjoint with each other, and this inequality
follows from the non-atomic property of \(B\) as we argue previously.
Otherwise, if
\(\sup \mu C <c - \mu\left(C_1 \cup C_2 \cup \ldots \cup C_{n-1}\right)\),
then denote
\(\delta = c - \mu\left(C_1 \cup C_2 \cup \ldots \cup C_{n-1}\right) - \sup \mu C\).
There will be no set \(C^\prime\) in
\(B-\left(C_1 \cup C_2 \ldots \cup C_{n-1}\right)\) satisfying
\(\mu C^\prime < \delta\), which leads to the same contraction in the
previous paragraph). Therefore, by the continuity,
\(\lim_{n\to\infty} \mu\left(\bigcup_{i=1}^n C_n\right) = \mu\left(\lim_{n\to\infty}\bigcup_{i=1}^n C_n\right) = c\).
Then \(\mu C=c\) for \(C=\bigcup C_{n .}\)

Can \(\mu\) be replaced by \(\varphi\)? Is it possible for \(\varphi\)
to take every value \(c\) between some values?

We may show that \(\varphi\) could take any value between \(-\varphi^-\)
and \(\varphi^+\). Since \(\varphi = \varphi^+ - \varphi^-\),
\(\varphi\) being an additive function, it can not take values outside
the aforementioned range within a measurable \(B \subset A\). By
Jordan-Hahn decomposition theorem, there exists \(BD \subset B\) such
that \(\varphi(BD) = \varphi^+(B)\) and \(BD^c \subset B\) such that
\(\varphi(BD^c) = -\varphi^-(B)\). So that \(\varphi\) could take values
at the edge. How about the in-between values? Again, by Jordan-Hahn
decomposition theorem, we know that \(\varphi+\) and \(\varphi^-\) are
measures, which means that, based on the conclusion for \(\mu\), for any
values \(c^+\) between \(0\) and \(\varphi^+(B)\), there exists
\(C^+ \subset B\) such that \(\varphi^+(C^+) = c^+\) and so
\(C^+D \subset B\) satisfies \(\varphi(C^+D) = c^+\). Similarly, we have
for any values \(c^-\) between \(\varphi^-(B)\) and 0 such that there is
\(C^- D \subset B\) and \(\varphi(C^-D) = c^-\). So the conclusion also
works for the signed measure. \(\square\)

\begin{center}\rule{0.5\linewidth}{0.5pt}\end{center}

\hypertarget{exercise-8-check}{%
  \subsubsection{Exercise 8 (Check)}\label{exercise-8-check}}

If \(\varphi\) is finitely additive, \(\mu\) is finite, and
\(\mu A_n \rightarrow 0\) implies \(\varphi A_n \rightarrow 0\), then
\(\varphi\) is \(\sigma\)-additive. \emph{We say that \(\varphi\) is
  \(\varphi_0\)-continuous if \(\varphi_0 A=0\) implies \(\varphi A=0\)}.

\textbf{Solution}

Consider a sequence of sets \(A_n \uparrow A\) for some \(A\) as
\(n\to \infty\). we know that \(B_i = A - A_i \to \empty\) as
\(n\to \infty\). Since \(\varphi\) is finitely addtive, for any \(n\) we
have

\[\varphi\left(\sum_{i=1}^n B_i \right) = \sum_{i=1}^n \varphi(B_i)\]

Now, using the same technique in the textbook, for any \(m < \infty\)

\[\varphi\left(\sum_{i=1}^\infty B_i\right) = \sum_{i=1}^m \varphi(B_i) + \varphi\left(\sum_{i = m+1}^\infty B_i\right).\]

Since \(\sum_{i = m+1}^\infty B_i = A - A_{m} \to \emptyset\),
\(\mu\left(\sum_{i = m+1}^\infty B_i\right) \to 0\) by the its
\(\sigma\)-additivity (by \emph{CONTINUITY THEOREM FOR ADDITIVE SET
  FUNCTIONS}). Hence, by assumption, we have
\(\varphi\left(\sum_{i = m+1}^\infty B_i\right) \to 0\) as
\(m\to \infty\). As a result,

\[\varphi\left(\sum_{i=1}^\infty B_i\right) = \lim_{m\to \infty} \sum_{i=1}^m \varphi(B_i) = \sum_{i=1}^\infty \varphi(B_i),\]

which means \(\sigma\)-addtivity. \(\square\)

\begin{center}\rule{0.5\linewidth}{0.5pt}\end{center}

\hypertarget{exercise-9-discussion}{%
  \subsubsection{Exercise 9 (Discussion)}\label{exercise-9-discussion}}

If \(\mu A_n \rightarrow 0\) implies
\(\varphi A_n \rightarrow 0\left(\bar{\varphi} A_n \rightarrow 0\right)\),
then \(\varphi\) is \(\mu\)-continuous. If \(\varphi\) is finite, then
the converse is true.\\
(Assume the contrary of the converse; there exist \(\epsilon>0\) and
\(A_n\) such that \(\mu A_n<\frac{1}{2^n}\) and
\(\bar{\varphi} A_n \geqq \epsilon\). Then \(\mu B=0\) and
\(\bar{\varphi} B \geqq \epsilon\) for \(B=\lim \sup A_n\).)\\
What if \(\varphi\) is \(\sigma\)-finite? What about \(\mathfrak{a}\)
consisting of all subsets of a denumerable space of points \(\omega_n\),
and
\(\mu\left\{\omega_n\right\}=\frac{1}{2^n}, \varphi\left\{\omega_n\right\}=n\).
What about \(\mu\) replaced by \(\varphi_0\) ?

\textbf{Solution}

Based on the previous definition, the contrary of the converse case is
that:

\emph{\(\mu A\) = 0 implies \(\varphi A = 0\) \(\nRightarrow\)
  \(\mu A_n \rightarrow 0\) implies \(\varphi A_n \rightarrow 0\)}

Then by the given hint, the contrary of the converse is possible, then
there exists \(\epsilon > 0\) and \(A_n \downarrow\) (We can always find
such a sequence. The most extreme case is that for all \(n\),
\(\mu A_n = 0\)) such that \(\mu A_n<\frac{1}{2^n}\) and
\(\varphi A_n \ge \epsilon\). Take
\(B = \lim \sup A_n = \cap_{n=1}^\infty\cup_{k=n}^\infty A_k\). Then
\(\mu B = \mu(\lim \sup A_n ) = 0\) by continuity. Since for any
\(\delta >0\), there is always an \(n_0 > 0\) such that
\(\frac{1}{2^{n_0-1}} < \delta\). Then,

\[\mu\left(\bigcup_{k=n_0}^\infty A_k\right) \le \sum_{k = n_0}^\infty \mu(A_k) \le \frac{1}{2^{n_0-1}} < \delta\]

As a result,
\(0 \le \mu(B) < \mu\left(\bigcup_{k = k_0}^\infty A_k\right)\) for any
\(k_0\) so \(\mu(B) = 0\).

On the other hand
\(\varphi(B) = \varphi(\lim \sup A_n) \ge \varphi(\lim A_n) = \lim \varphi A_n \ge  \epsilon\),
by continuity of the finite measure \(\varphi\). which is contradictory
to the assumption.

The given example is a case when \(\varphi\) is \(\sigma\)-finite. In
this case, the converse is not true. This is because if we let
\(A_n = \cup_{k = n}^\infty \{w_k\}\), then
\(\mu A_n = \sum_{k = n}^\infty \frac{1}{2^k} < \frac{1}{2^{n-1}} \to 0\)
as \(n\to \infty\) while \(\varphi A_n = \sum_{k=n}^\infty k = \infty\).
Yet, we still have \(\mu(\emptyset) = \varphi(\emptyset) = 0\).

What is the role of finiteness? With finiteness assumed, we will have
continuity from above at \(\emptyset\) for the monotone set sequence,
but for \(\sigma\)-finite measure we can\textquotesingle t guarantee it
because the measure of a monotone decreasing sequence could be always
infinity, yet the limit of the sets is the empty set!

If \(\mu\) is replaced by a signed measure \(\varphi_0\). Then we
consider the same argument by taking the absolute value of the signed
measure. \(\square\)

\begin{center}\rule{0.5\linewidth}{0.5pt}\end{center}

\hypertarget{exercise-10-check}{%
  \subsubsection{Exercise 10 (Check)}\label{exercise-10-check}}

If the \(\mu_j\) are finite measures, then there exists a \(\mu\) such
that all the \(\mu_j\) are \(\mu\)-continuous. (Take
\(\mu=\sum \mu_j / 2^j \mu_j \Omega\).) What about \(\mu_j\)
\textquotesingle s replaced by \(\varphi_j\) \textquotesingle s?

\textbf{Solution}

Take \(\mu=\sum \mu_j / 2^j \mu_j \Omega\). Since it is the linear
combination of \(\mu_j\)\textquotesingle s, it is a measure. If
\(\mu = 0\), by the nonnegativity of measures, we must have all
\(\mu_j\)\textquotesingle s being zero.

If \(\mu_j\)\textquotesingle s are replaced by signed measures
\(\varphi_j\)\textquotesingle s. Let\textquotesingle s consider
\(\varphi_j^+\) and \(\varphi_j^-\)\textquotesingle s. Following the
same manner we can construct \(\varphi^\pm\) such that
\(\varphi^\pm = 0 \implies \varphi_j^\pm = 0\). If we consider
\(\bar \varphi\), then \(\bar\varphi = 0\) iff \(\varphi^\pm = 0\).
Therefore,

\[\bar\varphi = 0\iff \varphi^\pm = 0 \implies \varphi_j^{\pm} = 0 \implies \varphi_j = 0\]

\(\square\)

\begin{center}\rule{0.5\linewidth}{0.5pt}\end{center}

Let \(\mathfrak B \subset \mathfrak{a}\) be a \(\sigma\)-field such that
the measurable subsets of elements of \(\mathfrak B\) belong to
\(\mathfrak B\). Let \(\mathfrak B(\varphi)\) be the class of sets such
that their subsets which belong to \(\mathfrak B\) are \(\varphi\)-null.
Call the sets of \(\mathfrak B\) "singular," and the sets of
\(\mathfrak B(\varphi)\) "regular." Call \(\varphi\) regular (singular)
if every singular (regular) set is \(\varphi\)-null.\\
Let
\(\varphi_r=\varphi_r^{+}-\varphi_r^{-}, \varphi_s=\varphi_s^{+}-\varphi_s^{-}\),
defined by

\begin{array}{ll}
  \varphi_r{ }^{ \pm}(A)=\sup \varphi^{ \pm}(B)    & \text { for all regular } \quad B \subset A,   \\
  \varphi_{\star}^{ \pm}(A)=\sup \varphi^{ \pm}(B) & \text { for all singular } \quad B \subset A .
\end{array}

\hypertarget{exercise-11-undone}{%
  \subsubsection{Exercise 11 (Undone)}\label{exercise-11-undone}}

\emph{Decomposition theorem}. \(\varphi_r\) is regular, \(\varphi_s\) is
singular, and \(\varphi=\varphi_r+\varphi_{\mathrm{s}}\). If \(\varphi\)
is finite, then the decomposition of \(\varphi\) into a regular and a
singular part is unique. What if \(\varphi\) is \(\sigma\)-finite? What
if \(\mathfrak a\) consists of all subsets of a noncountable space, and
\(\varphi(A)\) equals the number of points of \(A\) ? (Proceed as
follows:

\begin{enumerate}
  \def\labelenumi{\arabic{enumi}.}
  \item
        \(B(\varphi)=B(\bar{\varphi})=B\left(\varphi^{+}\right) \cap B\left(\varphi^{-}\right)\)is
        a \(\sigma\)-field.
  \item
        \(\varphi_r\left(\varphi_s\right)\) is a regular (singular) signed
        measure.
  \item
        Every \(A\) contains disjoint \(A_r\) regular and \(A_s\) singular
        such that \(\varphi_r \pm(A)\)
        \(=\varphi^{ \pm}\left(A_r\right), \varphi_s^{ \pm}(A)=\varphi^{ \pm}\left(A_s\right)\).
  \item
        If \(A=A^{\prime}{ }_r+A^{\prime}\), with \(A^{\prime}\), regular and
        \(A^{\prime}\), singular, then we can take \(A_r=A^{\prime}{ }_r\) and
        \(A_s=A^{\prime}{ }_s\).
  \item
        If \(\varphi\) is finite, every \(A\) can be so decomposed.)
\end{enumerate}

\textbf{Solution}

\hypertarget{exercise-12-undone}{%
  \subsubsection{Exercise 12 (Undone)}\label{exercise-12-undone}}

We can take for singular sets:

\begin{enumerate}
  \def\labelenumi{\arabic{enumi}.}
  \item
        the \(\mu\)-null sets-regular (singular) becomes \(\mu\)-continuous (
        \(\mu\)-discontinuous);
  \item
        the countable measurable sets-regular (singular) becomes continuous
        (purely discontinuous);
  \item
        the countable sums of atoms-regular (singular) becomes nonatomic
        (atomic).
\end{enumerate}

In each case investigate the regular and singular parts.

\begin{center}\rule{0.5\linewidth}{0.5pt}\end{center}

\hypertarget{exercise-13-discussion}{%
  \subsubsection{Exercise 13 (Discussion)}\label{exercise-13-discussion}}

\emph{Intermediate-value theorem} (compare with continuous function on a
connected set). If \(A\) is nonatomic and \(A_n \uparrow A\) with
\(\varphi A_n\) finite, then \(\varphi\) takes every value between
\(-\varphi^-A\) and \(+\varphi^{+} A\) for measurable subsets in \(A\).
(See 7.) What if \(\mathfrak a\) consists of all sets in a noncountable
space, \(\varphi(A)=0\) or \(\infty\) according as \(A\) is countable or
not?

\textbf{Solution}

For countable \(A\)\textquotesingle s the results is still the same,
\(\varphi(A)\) could only be \(0\). For uncountable
\(A\)\textquotesingle s, since they always have countable subsets, and
it is impossible for \(\varphi\) to take values other than \(0\) and
\(\infty\), the conclusion therefore does not hold.

\begin{center}\rule{0.5\linewidth}{0.5pt}\end{center}

In what follows, the \(\varphi_n\) are \(\sigma\)-additive but, unless
otherwise stated, \(\lim \varphi_n\) is not assumed to be
\(\sigma\)-additive.

\emph{Page 86 c. If \(\varphi\) on a \(\sigma\)-field \(\mathfrak a\) is
  \(\sigma\)-additive, then there exist sets \(C\) and \(D\) of \(a\) such
  that \(\varphi(C)=\sup \varphi\) and \(\varphi(D)=\inf \varphi\).}

\hypertarget{exercise-14}{%
  \subsubsection{Exercise 14}\label{exercise-14}}

If \(\varphi_n \rightarrow \varphi\) \(\sigma\)-addtive, then
\(\varphi ^\pm \le \lim \inf \varphi_n ^\pm\). If, moreover,
\(\varphi_n \uparrow\) or \(\varphi_n \downarrow\), then
\(\varphi^{ \pm}=\lim \varphi_n ^\pm\).

\textbf{Solution}

\(\forall A \in \mathfrak a\), recall that

\[\varphi^+(A) = \sup_{B\subset A}\varphi(B); \quad \varphi^-(A) = -\inf_{B\subset A} \varphi(B).\]

We first show for \(\varphi^+\). Let \(B_0 \subset A\) be such that
\(\varphi(B_0) = \varphi^+(A)\) and \(B_n \subset A\) such that
\(\varphi_n(B_n) = \varphi_n^+(A)\) (this is possible because of c. on
page 86, where \(\sigma\)-additivity plays the role,) then
\(\forall \epsilon > 0\), \(\exists N_1 > 0\) such that
\(\forall n > N_1\),

\[\varphi^+(A) = \varphi(B_0) < \varphi_n(B_0) + \frac{\epsilon}{2} \le \varphi_n(B_n)+ \frac{\epsilon}{2} = \varphi_n^+(A)+ \frac{\epsilon}{2}.\]

On the other hand, \(\forall N_1 > 0\), \(\exists n_2 > N_1\) such that

\[\lim\inf \varphi_{n}^+(A) + \frac{\epsilon}{2} > \varphi_{n_2}^+(A).\]

Therefore, \(\forall \epsilon > 0, A\in \mathfrak a\),

\[\varphi^+(A) < \lim\inf\varphi_n^+(A) + \epsilon \implies \varphi^+ \le \lim\inf \varphi^+_n.\]

Same techniques apply for \(\varphi^-\).

If \(\varphi_n\uparrow\), we know that \(\varphi \ge \varphi_n\) and so

\[\varphi^+_n(A) = \varphi_n(B_n) \le \varphi(B_n) \le \varphi(B_0) = \varphi^+(A).\]

\(\forall \epsilon > 0\), \(\forall N_1 > 0\), \(\exists n_3 > N_1\),
such that

\[\lim\sup \varphi_n^+(A) - \epsilon < \varphi_{n_3}^+(A)\le \varphi^+(A).\]

Therefore, \(\forall \epsilon > 0, A\in \mathfrak a\),

\[\varphi^+(A) > \lim\sup\varphi_n^+(A)  - \epsilon \implies \varphi^+ \ge \lim\sup\varphi_n^+.\]

To wrap up,

\[\lim\sup\varphi_n^+ \le \varphi^+\le \lim\inf_n\varphi^+ \implies \varphi^+ = \lim\varphi_n^+.\]

If \(\varphi_n \downarrow\), then
\(\forall \epsilon > 0, A\in\mathfrak a\), \(\exists N_4 > 0\), such
that \(\forall n > N_4\)

\[\varphi_n(A) < \varphi(A) + \frac{\epsilon}{2} \implies \varphi_n^+(A) = \varphi_n(B_n) < \varphi(B_n) + \frac{\epsilon}{2} \le \varphi(B_0) + \frac{\epsilon}{2}\]

On the other hand, \(\forall \epsilon > 0\), \(\forall N_4 > 0\),
\(\exists n_4 > N_4\), such that

\[\varphi_{n_4}^+(A) > \lim\sup \varphi_n^+(A) - \frac{\epsilon}{2}\]

The rest remains the same. Same techniques apply for \(\varphi^-\).
\(\square\)

\begin{center}\rule{0.5\linewidth}{0.5pt}\end{center}

\hypertarget{exercise-15-check}{%
  \subsubsection{Exercise 15 (Check)}\label{exercise-15-check}}

If \(\varphi_n \uparrow(\downarrow)\) and
\(\varphi_1>-\infty(<+\infty)\), then \(\varphi_n \rightarrow \varphi\)
\( \sigma\)-additive.

\textbf{Solution}

Consider \(\varphi_n \uparrow\). It is easier to show that \(\varphi\)
is \(\sigma\)-subadditive. \(\forall \epsilon > 0\), \(\exists N > 0\)
such that \(\forall n > N\), for arbitrary disjoint
\(A_k\)\textquotesingle s,

\begin{aligned}
  \varphi\left(\sum_{k=1}^\infty A_k\right) < \varphi_n\left(\sum_{k=1}^\infty A_k\right) + \epsilon = \sum_{k=1}^\infty \varphi_n\left(A_k\right) + \epsilon \le  \sum_{k=1}^\infty\varphi\left(A_k\right) + \epsilon
\end{aligned}

Therefore,
\(\varphi\left(\sum_{k=1}^\infty A_k\right) \le \sum_{k=1}^\infty\varphi\left(A_k\right)\).

On the other hand, by the monotonicity of \(\varphi_n\).
\(\forall \epsilon > 0\), there exists \(N_1\), such that
\(\forall n > N_1\),
\(\varphi_{n}(A_1) > \varphi(A_1) - \frac{\epsilon}{2}\)... In general,
we can always have \(N_k > N_{k-1} > \dots > N_1\),
\(\varphi_{N_k+1}(A_k) > \varphi(A_k) - \frac{\epsilon}{2^k}\).
Therefore,

\[\sum_{k=1}^K \varphi(A_k) - \epsilon < \sum_{k=1}^K\varphi_{N_k+1}(A_k) = \varphi_{N_K+1}\left(\sum_{k=1}^K A_k \right) \le \varphi\left(\sum_{k=1}^K A_k \right)\]

Therefore, \(\forall \epsilon > 0, K < \infty\),
\(\sum_{k=1}^K \varphi(A_k) - \epsilon < \varphi\left(\sum_{k=1}^K A_k \right) \implies \sum_{k=1}^K \varphi(A_k) \le \varphi\left(\sum_{k=1}^K A_k \right)\).
Since taking limit preserves partial inequality, we must have
\(\sum_{k=1}^\infty \varphi(A_k) \le \varphi\left(\sum_{i=1}^\infty A_k \right)\).

Hence,
\(\sum_{k=1}^\infty \varphi(A_k) = \varphi\left(\sum_{i=1}^\infty A_k \right)\).
For \(\varphi_n \downarrow\), let

\[M = \sup_{B \subset \sum_{i=1}^\infty A_k} \varphi_1(B)\]

Just consider \(\varphi_n^\prime = M - \varphi_n \uparrow\). \(\square\)

\begin{center}\rule{0.5\linewidth}{0.5pt}\end{center}

\hypertarget{exercise-16-check}{%
  \subsubsection{Exercise 16 (Check)}\label{exercise-16-check}}

If \(\varphi_n \rightarrow \varphi\) uniformly on \(\mathfrak a\) and
\(\varphi>-\infty\) or \(\varphi<+\infty\), then \(\varphi\) is
\(\sigma\)-additive.

\textbf{Solution}

The uniform convergence means that
\(\forall A\in \mathfrak a, \epsilon > 0\), \(\exists N > 0\) such that
when \(n > N\), \(|\varphi_n(A) - \varphi(A)| < \epsilon\).

For arbitrary disjoint \(A_k\)\textquotesingle s,
\(\forall \epsilon > 0, K \in \mathbb Z_+\), there exsits \(N_K > 0\)
such that when \(n > N_K\),

\[\varphi\left(\sum_{k=1}^K A_k\right) < \varphi_n\left(\sum_{k=1}^K A_k\right) + \frac{\epsilon}{K+1} = \sum_{k=1}^K\varphi_n\left( A_k\right) + \frac{\epsilon}{K+1} < \sum_{k=1}^K\varphi\left( A_k\right) + \epsilon,\]

and

\[\varphi\left(\sum_{k=1}^K A_k\right) > \varphi_n\left(\sum_{k=1}^K A_k\right) - \frac{\epsilon}{K+1} = \sum_{k=1}^K\varphi_n\left( A_k\right) - \frac{\epsilon}{K+1} > \sum_{k=1}^K\varphi\left( A_k\right) - \epsilon,\]

Therefore, \(\forall \epsilon > 0, \forall K \in \mathbb Z_+\)

\[\sum_{k=1}^K\varphi\left( A_k\right) - \epsilon < \varphi\left(\sum_{k=1}^K A_k\right) < \sum_{k=1}^K\varphi\left( A_k\right) + \epsilon \implies \sum_{k=1}^K\varphi\left( A_k\right) \le \varphi\left(\sum_{k=1}^K A_k\right) \le \sum_{k=1}^K\varphi\left( A_k\right)\]

which means that (the partial inequality is preserved)

\[\sum_{k=1}^\infty \varphi\left( A_k\right) \le  \varphi\left(\sum_{k=1}^\infty A_k\right)  \le \sum_{k=1}^\infty\varphi\left( A_k\right) \implies \varphi\left(\sum_{k=1}^\infty A_k\right) = \sum_{k=1}^\infty \varphi\left( A_k\right)\]

and \(\sigma\)-addtivity is proved. \(\square\)

\begin{center}\rule{0.5\linewidth}{0.5pt}\end{center}

\emph{If \(d\left(x_m, x_n\right) \rightarrow 0\) implies that
  \(x_n \rightarrow\) some \(x\), then the mutual convergence criterion is
  valid, and we say that the space is complete.}

\hypertarget{exercise-17-discussion}{%
  \subsubsection{Exercise 17 (Discussion)}\label{exercise-17-discussion}}

To a measure space \((\Omega, \mathfrak a, \mu)\) associate a complete
metric space \((\mathcal{X}, d)\) as follows: \(\mathcal X\) is the
space of all sets \(A, B\) of finite measure, \(d\) is a metric defined
by \(d(A, B)=\mu\left(A B^c+A^c B\right)\). Prove that the metric space
is complete.\\
(If \(A_n\) is a mutually convergent sequence in \(\mathcal X\), then
the sequence \(I_{A_n}\) mutually converges in measure and hence
converges in measure - see Page 116 6.3.)

If \(\nu\) on \(\mathfrak a\) is a finite \(\mu\)-continuous measure,
then \(\nu\) is defined and continuous on \((\mathcal{X}, d)\).

\textbf{Solution}

Consider a set sequence \(A_n\) such that \(\forall \epsilon > 0\),
\(\exists N > 0\), such that if \(n_1,n_2 > N\),
\(d(A_{n_1}, A_{n_2}) < \epsilon\). This means that

\[\mu\left(A_{n_1}A_{n_2}^c+A_{n_1}^cA_{n_2}\right) < \epsilon \implies \mu\left(\{x: x\in A_{n_1}\Delta A_{n_2}\}\right) < \epsilon \implies \mu\left(\left|I_{A_{n_1}} - I_{A_{n_2}}\right| > 0 \right) < \epsilon\]

This means, since convergence in measure and mutually convergence in
measure is equivalent, that there exists a set \(A\) such that

\[\mu\left(\left|I_{A_{n}} - I_{A}\right| > 0 \right) \to 0 \implies I_{A_{n_1}} \xrightarrow{\mu} I_A\]

Then \(\{x: x\in A_{n_1}\Delta A\} \to \emptyset\) and so
\(d(A_n, A) \to 0\). Hence \(A_n\to A\) and this metric space is
complete.

The last statement comes from that, by Exercise 9 (the converse is
true),

\[A_n \to A \implies d(A_n, A) \to 0 \implies \mu(A \Delta A_n) \to 0 \implies v(A \Delta A_n) \to 0 \implies v(A_n) \to v(A)\]

The last step follows from the fact that

\[A_n = A - (A - A_n) + (A_n - A)\]

and \(v(A_n\Delta A) = v(A - A_n) + v(A_n - A) \to 0\). \(\square\)

\begin{center}\rule{0.5\linewidth}{0.5pt}\end{center}

We say that the \(\varphi_n\) are uniformly \(\mu\)-continuous if
\(\mu A_m \rightarrow 0\) implies \(\varphi_n A_m \rightarrow 0\)
uniformly in \(n\), as \(m \rightarrow \infty\).

\emph{A set \(A\) is nowhere dense if the complement of \(\bar{A}\) is
  dense in the space, or, equivalently, if \(\bar{A}\) contains no
  spheres, that is, if the interior of \(\bar{A}\) is empty. A set is of
  the first category if it is a countable union of nowhere dense sets, and
  it is of the second category if it is not of the first category.}

\emph{\textbf{Baire\textquotesingle s category theorem} Every complete
  metric space is of the second category.}

\hypertarget{exercise-18-discussion}{%
  \subsubsection{Exercise 18 (Discussion)}\label{exercise-18-discussion}}

Let \(\mu\) be \(\sigma\)-finite. If the finite \(\varphi_n\) are
\(\mu\)-continuous and \(\lim \varphi_n\) exists and is finite, then the
\(\varphi_n\) are uniformly \(\mu\)-continuous and
\(\lim \varphi_n=\varphi\) is \(\mu\)-continuous and
\(\sigma\)-additive.

(For every \(\epsilon>0\), set
\(A_k=\bigcap_{m=k}^{\infty} \bigcap_{n=k}^{\infty}\left[A \in \mathcal X ;\left|\varphi_m A-\varphi_n A\right|\right.\left.\le \frac{\epsilon}{3}\right]\).
By (17), every \(A_k\) is closed. By Baire\textquotesingle s category
theorem, there exists \(k_0, d_0\) and \(A_0 \in \mathcal X\) such that
\(\left[A \in \mathcal X ; d\left(A, A_0\right)<d_0\right] \subset A_{k_0}\).
Let \(0<\delta_0<d_0\) such that \(\left|\varphi_n A\right|<\epsilon\)
whenever \(\mu A<\delta_0\) and \(n \le k_0\). If \(\mu A<\delta_0\),
then
\(d\left(A_0-A, A_0\right)<d_0,  d\left(A_0 \cup A, A_0\right)<d_0,\)
and
\(\left|\varphi_n A\right| \le\left|\varphi_{k_0} A\right|+\left.\left|\varphi_n\left(A_0 \cup A\right)-\varphi_{k_0}\left(A_0 \cup A\right)\right|+\left|\varphi_n\left(A_0-A\right)-\varphi_{k_0}\left(A_0-A\right)\right|.\right)\)

\textbf{Solution}

Why is every \(A_k\) closed?

Consider a sequence
\(\{B_{j}: B_j\in A, d(B_{j_1}, B_{j_2}) \to 0 \text{ as } j_1, j_2 \to \infty\}\).
This is a mutually convergence sequence in \(A_k\). We know from
Exercise 17 that this sequence has a limit in \(\mathcal X\), namely
\(B\) and \(d(B_j, B) \to 0\) as \(j \to \infty\). Now we need to show
that \(B\in A_k\), i.e., \(\forall m, n \ge k\),
\(|\varphi_m(B) - \varphi_n(B)| \le \frac{\epsilon}{3}\).

\(\forall \delta > 0\), based on the \(\mu\)-continuity, since
\(\varphi_n\)\textquotesingle s are finite, based on exercise 9,
\(\exists J\) such that if \(j > J\), then for given \(m, n > k\),
\(|\varphi_m(B_j) - \varphi_m(B)| < \frac{\delta}{2}\) and
\(|\varphi_n(B_j) - \varphi_n(B)|  < \frac{\delta}{2}\). This is because
the symmetric difference goes to \(\emptyset\) and \(\varphi_m\) as well
as \(\varphi_n\) are finite and \(\sigma\)-additive (so that we will
have continuity at \(\emptyset\)). Therefore,

\[|\varphi_m(B) - \varphi_n(B)| \le |\varphi_m(B_j) - \varphi_m(B)| + |\varphi_m(B_j) - \varphi_n(B_j)| + |\varphi_n(B_j) - \varphi_n(B)| = \frac{\epsilon}{3} + \delta.\]

Since \(\delta\) could be arbitrarily small, we must have
\(|\varphi_m(B) - \varphi_n(B)| \le \frac{\epsilon}{3}\). This means
that \(A_k\) is closed.

By Baire\textquotesingle s category theorem, there exists \(k_0, d_0\)
and \(A_0 \in \mathcal X\) such that
\(\left[A \in \mathcal X ; d\left(A, A_0\right)<d_0\right] \subset A_{k_0}\).
Why?

First of all, we know that \(\mathcal X\) contains a sphere, which is in
the form of
\(\widetilde{\mathcal X} = \left[A \in \mathcal X ; d\left(A, A_0\right) < d_0\right]\).
Now, we need to show that there exists \(k_0\) such that
\(\widetilde{\mathcal X} \subset A_{k_0}\). For any set
\(A \in \mathcal X\), given \(\varphi_n \to \varphi\), we know that
there exists some \(\tilde k\) such that whenever \(m, n > \tilde k\),

\[\left|\varphi_m\left(A\right) - \varphi_n\left(A\right)\right| < \frac{\epsilon}{3}.\]

Therefore, we have

\[\mathcal X = \bigcup_{k=1}^\infty A_k\]

As a result, we could select some
\(\widetilde{\mathcal X} \subsetneq \mathcal X =  \bigcup_{k=1}^\infty A_k\)
by choosing the proper value of \(d_0\), i.e., if for some \(d\),
\(\mathcal X = \left[A \in \mathcal X ; d\left(A, A_0\right) < d\right]\)
and for any \(\tilde d  < d\),
\(\mathcal X \supset \left[A \in \mathcal X ; d\left(A, A_0\right) < \tilde d\right]\),
then letting \(d_0 = \frac{d}{2}\) will suffice.
\(\widetilde{\mathcal X} \subsetneq \bigcup_{k=1}^\infty A_k\) means
that there exists some \(k_0\) such that
\(\widetilde{\mathcal X} \subsetneq \bigcup_{k=1}^{k_0} A_k\). By the
definition of \(A_k\) we know that \(A_{k+1} \supset A_k\). Hence,

\[\widetilde{\mathcal X} \subsetneq \bigcup_{k=1}^{k_0} A_k = A_{k_0}.\]

By \(\mu\)-continuity, \(\exists \delta_0\) such that when
\(\mu A(\text{ or some }A_m) < \delta_0\),
\(|\varphi_n A| < \frac{\epsilon}{3}\) where \(n \le k_0\) (finite
\(n\)\textquotesingle s). This could always be done based on exercise 9,
i.e., \(\mu \to 0\) implies \(\varphi_n \to 0\) because of finiteness.
If \(\mu A<\delta_0\), then

\[d\left(A_0-A, A_0\right) = \mu\left((A_0 - A)^c A_0 + (A_0 - A)A_0^c\right) = \mu(A_0\cap A) \le \mu(A) < \delta_0 < d_0,\]

\[d\left(A_0 \cup A, A_0\right) = \mu((A_0\cup A)^c A_0 + (A_0\cup A)A_0^c) = \mu(A - A_0) < \mu(A) < \delta_0 < d_0.\]

This means that \(A_0 - A\) and \(A_0 \cup A\) are elements in
\(A_{k_0}\).

And so,

\begin{aligned}
  \left|\varphi_n A\right| & = \left|\varphi_{k_0} A + \varphi_n A - \varphi_{k_0} A \right|                                                                                                                                                \\
                           & \le |\varphi_{k_0} A + \varphi_n (A_0 \cup A - (A_0 - A)) - \varphi_{k_0} (A_0 \cup A - (A_0 - A))|                                                                                                            \\
                           & \le\left|\varphi_{k_0} A\right|+\underbrace{\left|\varphi_n\left(A_0 \cup A\right)-\varphi_{k_0}\left(A_0 \cup A\right)\right|+\left|\varphi_n\left(A_0-A\right)-\varphi_{k_0}\left(A_0-A\right)\right|}_{C_0} \\
\end{aligned}

As we know, since \(|\varphi_n A| < \frac{\epsilon}{3}\), we must have
\(C_0 < \frac{\epsilon}{3}\) otherwise we will have negative
\(|\varphi_{k_0} A|\).

On the other hand

\begin{aligned}
  |\varphi_{k_0} A| & = |\varphi_n A + \varphi_{k_0}A - \varphi_n A|                                                                                                                                            \\
                    & \le\left|\varphi_{n} A\right|+\left|\varphi_n\left(A_0 \cup A\right)-\varphi_{k_0}\left(A_0 \cup A\right)\right|+\left|\varphi_n\left(A_0-A\right)-\varphi_{k_0}\left(A_0-A\right)\right| \\
                    & \le \epsilon + C_0 \le \frac{2\epsilon}{3}
\end{aligned}

Eventually, \(t > k_0\), since \(A_0 - A\) and \(A_0 \cup A\) are
elements in \(A_{k_0}\),

\begin{aligned}
  \left|\varphi_n\left(A_0 \cup A\right)-\varphi_{t}\left(A_0 \cup A\right)\right| & \le \left|\varphi_t\left(A_0 \cup A\right)-\varphi_{k_0}\left(A_0 \cup A\right)\right| + \left|\varphi_n\left(A_0 \cup A\right)-\varphi_{k_0}\left(A_0 \cup A\right)\right| \\
                                                                                   & \le \left|\varphi_n\left(A_0 \cup A\right)-\varphi_{k_0}\left(A_0 \cup A\right)\right| + \frac{\epsilon}{3}
\end{aligned}

Similarly, we will have

\[\left|\varphi_n\left(A_0-A\right)-\varphi_{t}\left(A_0-A\right)\right|  \le \left|\varphi_n\left(A_0-A\right)-\varphi_{k_0}\left(A_0-A\right)\right| + \frac{\epsilon}{3}\]

As a result,

\[|\varphi_t A| \le |\varphi_n A|+ C_0 + \frac{\epsilon}{3} \le \frac{4}{3}\epsilon\]

Hence, as long as \(\mu A < \delta_0\) we will have \(\forall n > 0\),
\(\varphi_n A \le \frac{4}{3}\epsilon \). This means uniform
\(\mu\)-continuity. The definition of \(A_{k_0}\) guarantees that
\(|\varphi A| \le \frac{4}{3}\epsilon\) as well because the argument
would be the same as that for \(\varphi_t\) based on the fact that
taking the limit will preserve the inequality condition for \(A_{k_0}\).
As a result, we prove for the \(\mu\)-continuity for \(\varphi\) as
well.

By exercise 8, we know that if \(\mu\) is finite, \(\varphi\) being
\(\mu\)-continuous and finitely additive, then \(\varphi\) is
\(\sigma\)-additive. The finite addtivity is obvious since the limit of
the finite sum is the finite sum of the limit. On the other hand, in
this exercise, we know that \(\mu\) is \(\sigma\)-finite, which means
that if we ignore trivialities, there is at least one finite value for
\(\mu\). This means that \(\mu(\emptyset) = 0\). Based on the continuity
of \(\sigma\)-additive function, i.e.,
\(\mu(\lim A_n) = \lim \mu(A_n)\), and using the same argument of
exercise 8, we can claim that \(\varphi\) is \(\sigma\)-additive. Here
we don\textquotesingle t have the special case in exercise 9 since
\(\varphi\) is guaranteed to be finite.

\(\square\)

\begin{center}\rule{0.5\linewidth}{0.5pt}\end{center}

\hypertarget{exercise-19}{%
  \subsubsection{Exercise 19}\label{exercise-19}}

If finite \(\varphi_n \rightarrow \varphi\) finite, then \(\varphi\) is
\(\sigma\)-additive. (If \(\left|\varphi_n\right| \le c_n\), set
\(\mu A=\sum \frac{1}{2^n c_n}\left|\varphi_n A\right|\) and apply 18.)

\textbf{Solution}

If \(\left|\varphi_n\right| \le c_n\), set
\(\mu A=\sum \frac{1}{2^n c_n}\left|\varphi_n A\right|\) . Then
\(\varphi_n\)\textquotesingle s are \(\mu\)-continuous, i.e.
\(\mu A = 0 \implies \varphi_n A = 0\). By exercise 18, \(\varphi_n\)
are uniformly \(\mu\)-continuous and \(\lim \varphi_n=\varphi\) is
\(\mu\)-continuous and \(\sigma\)-additive. \(\square\)

\hypertarget{chapter-2-exercise}{%
  \subsection{Chapter 2 Exercise}\label{chapter-2-exercise}}

\textbf{Notation.} Unless otherwise stated, the measure space
\((\Omega,\mathfrak a, \mu)\) is fixed, the (measurable) sets
\(A, B, ...,\) with or without affixes, belong to \(\mathfrak a\), and
the functions \(X, Y, \dots\) with or without affixes, are finite
measurable functions.

\hypertarget{exercise-1}{%
  \subsubsection{\texorpdfstring{Exercise 1
    }{Exercise 1 }}\label{exercise-1}}

The set \(C\) of convergence of a sequence \(X_{n}\) (to a finite or
inlinite limit function) is measurable.

\[(C=\left[\operatorname*{lim}\,\operatorname{inf}X_{n}=\operatorname*{lim}\,\operatorname*{sup}\,X_{n}\right].)\]

\textbf{Solution}

To be specific, we could write

\[C = \left[\omega\in \Omega: \lim\inf X_n(\omega) = \lim\sup X_n(\omega)\right]\]

Consider the function \(Y_1 = \lim \inf X_n\) and
\(Y_2 = \lim\sup X_n\), then according to the \emph{MEASURABILITY
  THEOREM}, they are measurable. If we define \(Y = Y_1 - Y_2\), then it
is the result of an arithmetic operation of \(Y_1\) and \(Y_2\) so that
it is also measurable. Notice that

\[C = Y^{-1}(0).\]

Since it is a inverse image of a Borel set \(\{0\}\) in \(\mathbb R\),
\(C\) is a measurable set. \(\square\)

\begin{center}\rule{0.5\linewidth}{0.5pt}\end{center}

A countably valued function \(X=\sum x_j I_{A_j}\) where the sets
\(A_j\) are measurable is called an elementary measurable function or,
simply, an \emph{elementary function}; if the number of distinct values
of \(X\) is finite, then \(X\) is also called a \emph{simple function}.

\hypertarget{exercise-2-2}{%
  \subsubsection{\texorpdfstring{Exercise 2
    }{Exercise 2 }}\label{exercise-2-2}}

If \(\mu\) is finite, then given \(X\), for every \(\epsilon > 0\) there
exists \(A\) such that \(\mu A<\epsilon\) and \(X\) is bounded on
\(A^{c}.\) If \(X\) is bounded, then there exists a sequence of simple
functions which converges uniformly to \(X.\) Combine both propositions.

We say that a sequence \(X_n\) converges almost uniformly (a.u.) to
\(X\), and write \(X_n \stackrel{\text { a.u. }}{\longrightarrow} X\),
if, for every \(\epsilon>0\), there exists a set \(A\) with
\(\mu A<\epsilon\) such that
\(X_n \stackrel{\mathrm{u}}{\longrightarrow} X\) on \(A^c\).

\textbf{Solution}

For the first statement, since \(\mu\) is finite, we have that
\(\mu(\emptyset) = 0\). If the statement is not true, then there exists
\(\epsilon_0 > 0\) such that there does not exist a set \(A\) satisfying
\(\mu A < \epsilon_0\) and \(X\) is bounded on \(A^c\). (Notice that
finite does not mean bounded, consider \(f(x) = x\). )

There are two possible situations, the first is that
\(\forall B \ne \emptyset, B \in \mathfrak a\), \(\mu B \ge \epsilon_0\)
or \(\mu B = 0\). Since \(\mu\) is finite, we can write
\(\mu \Omega = M < \infty\). This means that the number of disjoint sets
such that its measure is greater than \(\epsilon_0\) is at most
\(\frac{M}{\epsilon_0}\), which is a finite number. For any one these
set, say \(C\), it is possible to claim that for each single element
subset \(\{\omega\} \subset C\) is either \(\mu\)-null or with
\(\mu\)-value no less than \(\epsilon_0\). This further leads to the
fact that there are at most \(\frac{M}{\epsilon_0}\) single element set
with \(\mu\)-value no less than \(\epsilon_0\)subsets. It is possible to
find the maximum of \(|X|\), say \(x_0\), over these finite number of
single elements and so on a set with \(\mu\)-value \(M\), we have
\(|X| < x_0\), which means the statement is actually true.

The second possible situation is the one that does not belong to the
first. If the first statement is not true, then for any \(x_0 > 0\),
there exists \(\epsilon_0\) such that
\(\mu(A_0 = \{\omega: |X(\omega)| \le x_0\}) <= M - \epsilon_0\). But
this means that over a non-empty set \(A_0\), for any \(x_0 > 0\),
\(|X| > x_0\) which means \(|X| = \infty\) and it is contradictory.
Therefore, the first statement has to be true.

For the second statement, this is proposition
C\textquotesingle\textquotesingle{} of measurability theorem. Since
\(X\) is bounded, we can assume that \(|X| < x_0\). On the other hand,
since \(X\) is a measurable function, the set
\(\{\omega:  \frac{k-1}{2^n} \le X(\omega) < \frac{k}{2^n}\}\) is
measurable for any \(n\in \mathbb N\) and \(k \in \mathbb Z\)(, and so
the indicator function is measurable as well). As a result, we can
consider a sequence of simple function \(\{X_n\}\) such that

\[X_n = \sum_{k = \lceil -x_02^n\rceil }^{\lceil x_0 2^n \rceil } \frac{k-1}{2^n} \mathbb I_{\{\frac{k-1}{2^n} \le X < \frac{k}{2^n}\}}.\]

It is true that \(|X_n - X| < \frac{1}{2^n}\) everywhere, which means
that \(X_n \xrightarrow{u} X\).

Finally we could see that it is very natural to define almost-uniform
convergence. \(\square\)

\begin{center}\rule{0.5\linewidth}{0.5pt}\end{center}

\hypertarget{exercise-3-2}{%
  \subsubsection{Exercise 3}\label{exercise-3-2}}

If \(X_n \stackrel{\text { a.u. }}{\longrightarrow} X\), then
\(X_n \stackrel{\text { a.e. }}{\longrightarrow} X\) and
\(X_n \stackrel{\mu}{\longrightarrow} X\). (For the first assertion,
form \(A_n\) where \(A_n\) is the \(A\) of the foregoing definition with
\(\epsilon=\frac{1}{n}\).)

\textbf{Solution}

For the first assertion, we consider a sequence of set \(\{A_n\}\) where
\(\mu(A_n) < \frac{1}{n}\) and on \(A_n^c\) we have
\(X_n\xrightarrow{u} X\) and so \(X_n \to X\). As a result, if we define
\(A = \cap_{n=1}^\infty A_n\), then on \(A^c\) we have \(X_n\to X\) .
However, if we consider the measure of \(A\) we will see that

\[\mu A = \mu\left( \bigcap_{n=1}^\infty A_n \right) \le \inf \mu(A_n) = 0.\]

This means that \(X_n \xrightarrow{\text{a.e.}} X\).

We don\textquotesingle t have \(\mu\) being finite, so we cannot use the
Comparison of Convergence Theorem. We may need to proof for convergence
in measure directly. Notice that for any \(\delta > 0\),
\(\epsilon > 0\), there exists \(A\) with \(\mu(A) < \epsilon\) and
\(X_n \xrightarrow{u} X\) on \(A^c\). This means that
\(\exist N_\delta\) such that when \(n > N_{\delta}\),

\[\mu\left(|X_n - X| > \delta\right) \le \mu(A) < \epsilon \implies \forall \delta,\ \mu\left(|X_n - X| > \delta\right) \to 0\]

This is just the definition of \(X_n \xrightarrow{\mu} X\). \(\square\)

\begin{center}\rule{0.5\linewidth}{0.5pt}\end{center}

\hypertarget{exercise-4-2}{%
  \subsubsection{Exercise 4}\label{exercise-4-2}}

If \(X_n \stackrel{\mu}{\longrightarrow} X\), then there exists a
subsequence
\(X_{n^{\prime}} \stackrel{\text { a.u. }}{\longrightarrow} X\).

\textbf{Solution}

Based on the definition of convergence by measure, we know that there
exists \(N_1\) such that when \(n_1 > N_1\),

\[\mu\left(|X_{n_1} - X| \ge 1 \right) := \mu(A_1) < 2^{-1}\]

Similarly, we can have \(n_2 > \max\{N_2, n_1\}\) such that

\[\mu\left(|X_{n_2} - X| \ge \frac{1}{2} \right):= \mu(A_2) < 2^{-2}.\]

In general, for \(k \in \mathbb N\), \(n_k > \max\{N_{k}, n_{k-1}\}\)
and

\[\mu\left(|X_{n_k} - X| \ge \frac{1}{k} \right):= \mu(A_k) < 2^{-k}.\]

In this way, we construct a sequence
\(\{X_{n^\prime}\} = \{X_{n_1}, X_{n_2}, \dots\}\). Now for any
\(\epsilon > 0\), let
\(N_{\epsilon} = \lceil 1 - \log_2 \epsilon\rceil + 1\), then consider
the set

\[A_{\epsilon} = \bigcup_{n^\prime=N_{\epsilon}}^\infty A_{n^\prime}, \ \mu(A_{\epsilon}) \le \sum_{n^\prime = N_\epsilon}^\infty \mu (A_{n^\prime}) < \sum_{i = 1}^\infty \frac{\epsilon}{2^{-i}} < \epsilon.\]

\(\forall \delta > 0\), on \(A_\epsilon^c\), we have that as long as
\(n^\prime > \max\left\{\left\lceil \frac{1}{\delta}\right\rceil + 1, N_{\epsilon}\right\}\),

\[|X_{n^\prime} - X| < \delta \implies X_{n^\prime} \xrightarrow{\text u} X.\]

Then this is the subsequence we are looking for.

\begin{center}\rule{0.5\linewidth}{0.5pt}\end{center}

\hypertarget{exercise-5-2}{%
  \subsubsection{Exercise 5}\label{exercise-5-2}}

\emph{Egoroff\textquotesingle s theorem.} If \(\mu\) is finite, then
\(X_n \stackrel{\text { a.e. }}{\longrightarrow} X\), implies that
\(X_n \stackrel{\text { a.u. }}{\longrightarrow} X\). Compare with 3.
(Neglect the null set of divergence, and form
\(A=\bigcup_{m=1}^{\infty} A_m\) with
\(A_m=\bigcup_{k \ge n(m)}\left[\left|X_k-X\right| \ge \frac{1}{m}\right]\)
and \(n(m)\) such that \(\mu A_m<\frac{\epsilon}{2^m}\).)

\textbf{Solution}

If \(\mu\) is finite, then by the convergence a.e. criterion (page 116),
we have

\[\mu \bigcup_\nu\left[\left|X_{n+\nu}-X\right| \ge \epsilon\right] \rightarrow 0\]

Therefore, \(\forall \epsilon\), based on the given hint, for any
\(m \in \mathbb N\), \(\exists n(m)\) such that when
\(k = n + v \ge n(m)\),

\[\mu \left(\bigcup_{k \ge n(m)}\left[\left|X_{k}-X\right| \ge \frac{1}{m}\right]\right) := \mu (A_m) < \frac{\epsilon}{2^m}\]

As a result, if we define \(A=\bigcup_{m=1}^{\infty} A_m\), then
\(\mu(A) < \epsilon\). Furthermore, on \(A^c\), \(\forall \delta > 0\),
as long as
\(n \ge n\left(\left\lceil \frac{1}{\delta}\right\rceil\right)\), we
have

\[|X_n - X| < \frac{1}{\left\lceil \frac{1}{\delta}\right\rceil} < \delta \implies X_n \xrightarrow{\text u} X\]

Combine the previous statement together, we have
\(X_n \xrightarrow{\text{a.u.}} X\).

The finiteness gives us a nice property that gets rid of the
intersection regarding \(n\). Actually, finiteness will give us
continuity from above, which further ensures that conditions in (70)
could be satisfied.

Notice that

\[\bigcap_n \bigcup_\nu\left[\left|X_{n+\nu}-X\right| \geqq \epsilon\right] = \lim_{n\to \infty }\bigcup_\nu\left[\left|X_{n+\nu}-X\right| \geqq \epsilon\right]\]

becaues
\(\bigcup_\nu\left[\left|X_{n+\nu}-X\right| \geqq \epsilon\right]\) is a
monotonic decreasing sequence as \(n \uparrow\). Therefore, the a.e.
criterion is equivalent to say that when \(\mu\) is finite,

\[\mu\left(\lim_{n\to \infty} \bigcup_\nu\left[\left|X_{n+\nu}-X\right| \geqq \epsilon\right]\right) = 0 \implies \lim \mu \left(\bigcup_\nu\left[\left|X_{n+\nu}-X\right| \geqq \epsilon\right]\right) \to 0.\]

which is true. We can also refer to exercise 9 of Chapter 1 for this.
\(\square\)

\begin{center}\rule{0.5\linewidth}{0.5pt}\end{center}

\hypertarget{exercise-6-2}{%
  \subsubsection{Exercise 6}\label{exercise-6-2}}

\emph{Lusin\textquotesingle s theorem.} If \(\mu\) is \(\sigma\)-finite,
then \(X_n \stackrel{\text { a.e. }}{\rightarrow} X\) implies that
\(X_n \stackrel{\mathrm{u}}{\rightarrow} X\) on every element \(A_j\) of
some countable partition of \(\Omega-N\) where \(N\) is some null set.
(Neglect the null set of divergence, and start with \(\mu\) finite. Use
Egoroff\textquotesingle s theorem to select inductively sets \(A_k\)
such that \(\mu \bigcap_{k=1}^n A_k<\frac{1}{n}\) and
\(X_n \stackrel{\mathrm{u}}{\rightarrow} X\) on \(A_k^c\) for every
\(\left.k.\right)\)

\textbf{Solution}

According to the hint, we start with \(\mu\) finite and we only need to
consider this case. In fact, \(\mu\) being \(\sigma\)-finite means that
we can partition \(\Omega\) into countably many
\(A_j\)\textquotesingle s where we have finite value of \(\mu\). Since
\(X_n \xrightarrow{\text{a.e.}} X\) and \(\mu\) is finite, we know from
Egoroff\textquotesingle s theorem that
\(X_n\xrightarrow{\text{a.u.}} X\). Therefore, we can select set
\(A_k, k = 1, 2, \dots\) as such that \(\mu(A_k) < \frac{1}{k}\) and on
\(A_k^c\) we have \(X_n \xrightarrow{\text{u}} X\). Now consider
\(A = \cap_{k=1}^\infty A_k\). We must have that for any \(n > 0\),

\[0 \le \mu(A) \le \inf \mu(A_k) < \frac{1}{n} \implies \mu(A) = 0\]

Therefore, we have that, except for a null set,
\(X_n\xrightarrow{\text u} X\). Since the countable union of null sets
is still null, the general conclusion for \(\Omega-N\) holds.

\begin{center}\rule{0.5\linewidth}{0.5pt}\end{center}

\hypertarget{exercise-7}{%
  \subsubsection{Exercise 7}\label{exercise-7}}

If \(\mu\) is finite, then
\(X_n \stackrel{\text { a.e. }}{\longrightarrow} X\) implies existence
of a set of positive measure on which the \(X_n\) are uniformly bounded.
What if \(\mu\) is \(\sigma\)-finite?

\textbf{Solution}

Based on exercise 2, if \(\mu\) is finite, suppose that
\(\mu(\Omega) =  \mu_0\), then
\(\exists A, 0 < \epsilon_0 < \frac{\mu_0}{4}\),
\(\mu(A) < \epsilon_0\), and on \(A^c\), there is \(x_0 < \infty\) such
that \(|X| < x_0\). Since \(X_n \xrightarrow{\text{a.e.}} X\), according
to the convergence a.e. criterion, we have that

\[\mu \bigcup_\nu\left[\left|X_{n+\nu}-X\right| \ge \epsilon\right] \rightarrow 0.\]

Therefore, there exists \(N > 0\) such that when \(n > N\),

\[\mu \bigcup_\nu\left[\left|X_{n+\nu}-X\right| \ge x_0 \right]:= \mu(A^\prime) < \frac{\mu_0}{4}.\]

As a result, we have that as long as \(n > N\), \(|X_n| < 2x_0\). For
the remaining \(N\) elements, \(X_1, \dots, X_N\), using the result of
exercise 2 again, there exists \(A_1, A_2, \dots, A_N\), all with
measure \(\frac{\mu_0}{4N}\) and on \(A_i^c\), \(X_i\) is bounded. Let
us denote

\[x_1 = \max\left\{2x_0, \max\left\{X_i(\omega): i=1,\dots,N, \omega\in \bigcup_{i=1}^N A_i\right\}\right\}.\]

If we let

\[B = \Omega - A - A^\prime - \bigcup_{i=1}^N A_i\]

then on \(B\), \(|X_n| < x_1\), and

\[\mu(B) > \mu_0 - \epsilon_0 - \frac{\mu_0}{4} - N\cdot\frac{\mu_0}{4N} > \frac{\mu_0}{4} = 0.\]

Therefore, \(B\) is the set we are looking for. If \(\mu\) is
\(\sigma\)-finite, then just look at one of its finite partition and
everything will be the same. \(\square\)

\begin{center}\rule{0.5\linewidth}{0.5pt}\end{center}

\hypertarget{exercise-8}{%
  \subsubsection{Exercise 8}\label{exercise-8}}

If \(\mu\) is finite, then
\(X_{m n} \stackrel{\text { a.e. }}{\longrightarrow} X_m\) as
\(n \rightarrow \infty\) and
\(X_m \stackrel{\text { a.e. }}{\longrightarrow} X\) as
\(m \rightarrow \infty\) imply that there exists subsequences
\(m_k, n_k\) such that
\(X_{m_k n_k} \stackrel{\text { a.e. }}{\longrightarrow} X\) as
\(k \rightarrow \infty\). What if \(\mu\) is \(\sigma\)-finite?\\
(Neglect the null sets of divergence. Select \(A_k\) and \(m_k\) such
that \(\mu A_k<\frac{1}{2^k}\) and
\(\left|X_{m_k}-X\right|<\frac{1}{2^k}\) on \(A_k^c\). Select
\(B_k \subset A_k\) and \(n_k\) such that \(\mu B_k<\frac{1}{2^k}\) and
\(\left|X_{m_k n_k}-X_{m_k}\right|<\frac{1}{2^k}\) on \(A_k-B_k\).)

\textbf{Solution}

If \(\mu\) is finite, then using egoroff\textquotesingle s theorem, we
know that \(X_{mn} \xrightarrow{\text{a.u.}} X_m\) and
\(X_m\xrightarrow{\text{a.u.}} X\). Now we begin the construction, for
\(k = 1, 2, \dots\), \(\exists A_k\) such that
\(\mu(A_k) < \frac{1}{2^{k+1}}\) and \(X_m \xrightarrow{\text{a.u.}} X\)
on \(A_k^c\). Therefore, \(\exists M_1 >0\), when \(m > M_1\), on
\(A_1^c\), \(|X_m - X| < \frac{1}{2^2}\). Take \(X_{m_1} = X_{M_1+1}\).
\(\exists M_2 >0\), when \(m_2 > M_2\), on \(A_2^c\),
\(|X_m - X| < \frac{1}{2^3}\). Take
\(X_{m_2} = X_{\max\{m_1+1, M_2+1\}}\), etc. So we have selected \(A_k\)
and \(m_k\) such that \(\mu A_k<\frac{1}{2^{k+1}}\) and
\(\left|X_{m_k}-X\right|<\frac{1}{2^{k+1}}\) on \(A_k^c\).

Similarly, we can select \(B_k\) and \(n_k\) such that
\(\mu B_k<\frac{1}{2^{k+1}}\) and
\(\left|X_{m_k n_k}-X_{m_k}\right|<\frac{1}{2^{k+1}}\) on \(B_k^c\). As
a result, we know that on \(C_k^c = A^c_k \cap B_k^c\),
\(\left|X_{m_k n_k}-X\right|<\frac{1}{2^{k}}\) by the triangular
inequality. And we can derive that

\[\mu\left(C_k\right) = \mu\left((A^c_k \cap B_k^c)^c\right) = \mu(A_k \cup B_k) < \frac{1}{2^k}\]

As a result, since \(\mu\) is finite, we can use the a.e. criterion,
\(\forall \epsilon > 0, \delta > 0\), let
\(k = \left\lceil 1 - \log_2\left(\min\{\epsilon, \delta\}\right)\right\rceil\)
+ 1, then

\begin{aligned}
  \mu\left(\bigcup_{v=0}^\infty \left[\left|X_{m_kn_k+v} - X\right|\ge \epsilon \right] \right)
   & \le \mu \left(\bigcup_{v=0}^\infty \left[\left|X_{m_kn_k+v} - X\right|\ge \frac{1}{2^{k+v}} \right] \right) \\
   & = \sum_{v = 0}^{\infty} \frac{1}{2^{k+v}} = \frac{1}{2^{k-1}} < \delta
\end{aligned}

This is equivalent to say that

\[\mu\left(\bigcup_{v=0}^\infty \left[\left|X_{m_kn_k+v} - X\right|\ge \epsilon \right] \right), \ k \to \infty\]

And so \(X_{m_kn_k}\xrightarrow{\text{a.e.}} X\).

If \(\mu\) is \(\sigma\)-finite, we can prove the same conclusion on
every \(\mu\)-finite partitions and then combine them together. Since
the countable union of null set is still null, the a.e. condition holds
for the universe \(\Omega\) as well. \(\square\)

\begin{center}\rule{0.5\linewidth}{0.5pt}\end{center}

\hypertarget{exercise-9-undone}{%
  \subsubsection{Exercise 9 (Undone)}\label{exercise-9-undone}}

Let
\(X_n \stackrel{\mu}{\rightarrow} X, Y_n \stackrel{\mu}{\rightarrow} Y\).
Do
\(a X_n+b Y_n \stackrel{\mu}{\rightarrow} a X+b Y,\left|X_n\right| \stackrel{\mu}{\rightarrow}|X|, X_n{ }^2 \stackrel{\mu}{\rightarrow}\)
\(X^2, X_n Y_n \stackrel{\mu}{\rightarrow} X Y\) ? What about
\(1 / X_n\) ? Let \(\mu\) be finite and let \(g\) on \(\mathbb R\) or on
\(\mathbb R \times \mathbb R\) be continuous. What about the sequences
\(g\left(X_n\right)\) and \(g\left(X_n, Y_n\right)\) ?

\textbf{Solution}

This is, in some sense, a combination of Slutsky Thoerem and continuous
mapping theorem... I don\textquotesingle t want to reinvent the wheel.

\begin{center}\rule{0.5\linewidth}{0.5pt}\end{center}

\hypertarget{exercise-10-undone}{%
  \subsubsection{Exercise 10 (Undone)}\label{exercise-10-undone}}

Let the functions \(X_n, X\) on the measure space be complex-valued or
vector-valued or, more generally, let them take their values in some
fixed Banach space. Denote the norm of \(X\) by \(|X|\), and denote
\(\left|X_n-X\right| \rightarrow 0\) by \(X_n \rightarrow X\).

Transpose the constructive definitions of measurability and the
definitions of various types of convergence. Investigate the validity of
the transposed of the corresponding properties established in the text,
as well as of those stated above.

\textbf{Solution}

They work. Not interesting.

\begin{center}\rule{0.5\linewidth}{0.5pt}\end{center}

\hypertarget{exercise-11}{%
  \subsubsection{Exercise 11}\label{exercise-11}}

Examples and counterexamples of mutual implications of types of
convergence. Investigate convergences of the sequences defined below:

\begin{enumerate}
  \def\labelenumi{\arabic{enumi}.}
  \item
        The measure space is the Borel line with Lebesgue measure, \(X_n=1\)
        on \([n, n+1]\) and \(X_n=0\) elsewhere.
  \item
        The measure space is the Borel interval \((0,1)\) with Lebesgue
        measure, \(X_n=1\) on \(\left(0, \frac{1}{n}\right)\) and \(X_n=0\)
        elsewhere.
  \item
        The measure space is the Borel interval \([0,1]\) with Lebesgue
        measure, the sequence is
        \(X_{11}, X_{21}, X_{22}, X_{31}, X_{32}, X_{33}, \cdots\) with
        \(X_{n k}=1\) on \(\left[\frac{k-1}{n}, \frac{k}{n}\right]\) and
        \(X_{n k}=0\) elsewhere.
  \item
        \(\mathfrak a\) consists of all subsets of the set of positive
        integers, \(\mu A\) is the number of points of \(A, X_n\) is indicator
        of the set of the \(n\) first integers.
\end{enumerate}

\textbf{Solution}

\textbf{For 1}, the limiting function could only be that \(X = 0\).
Therefore, it is not convergence in measure (the measure for deviating
from 0 will always be 1). If we consider a.e. convergence, then we
should investigate

\begin{aligned}
  \mu \bigcap_n \bigcup_\nu\left[\left|X_{n+\nu}-X\right| \geqq \epsilon\right]
   & = \mu \bigcap_n \bigcup_\nu\left[\left|X_{n+\nu}\right| \geqq \epsilon\right] \\
   & = \mu \bigcap_n [n, \infty)                                                   \\
   & = 0
\end{aligned}

This is that for any \(r < \infty, r\in \mathbb R\), there always exists
\(n > r\) so that \(r\) is not in the intersection. Therefore,
\(X_n \xrightarrow{\text{a.e.}} X\).

\textbf{For 2}, the limiting function should be \(X = 0\) and for any
\(\epsilon > 0, \delta > 0\), we know that as long as
\(n > \frac{1}{\delta}\),

\[\mu\left(\left[ |X_n - 0|\right] > \epsilon \right) \le \frac{1}{n} < \delta.\]

Hence, \(X_n \xrightarrow{\mu} X\).

On the other hand, since \(\mu\) is finite on \((0, 1)\), we can use a
simplified a.e. criterion. \(\forall \epsilon\), consider

\begin{aligned}
  \mu \bigcup_\nu\left[\left|X_{n+\nu}-X\right| \geqq \epsilon\right]
   & = \mu\left(\left(0, \frac{1}{n}\right)\right) \to 0, \text{ as }n \to \infty.
\end{aligned}

Hence, \(X_n \xrightarrow{\text{a.e.}} X\) as well.

\textbf{For 3}, the limiting funtion here is still \(X = 0\). The
sequence \(X_n \xrightarrow{\mu} X\) since
\(\forall \epsilon, \delta > 0\), as long as \(n > \frac{1}{\delta}\),

\[\mu\left(\left[ |X_n - 0|\right] > \epsilon \right) \le \frac{1}{n} < \delta.\]

\(\mu\) is still finite on \([0, 1]\) so we can consider the simplied
criterion for a.e. convergence. Notice that

\begin{aligned}
  \mu \bigcup_\nu\left[\left|X_{n+\nu}-X\right| \geqq \epsilon\right]
   & = \mu\left(\left[0, 1\right]\right) = 1 \not\to 0, \text{ as }n \to \infty,
\end{aligned}

We have no a.e. convergence here.

\textbf{For 4}, the limiting function \(X\) could be either \(0\) or
\(1\). For any finite \(n\),
\(\mu\left(|X_n - 1| > \epsilon\right) > \infty - n\) and
\(\mu\left(|X_n| > \epsilon\right) = n\), this indicates the failure of
convergence in measure. For a.e. convergene, if \(X = 1\), then

\begin{aligned}
  \mu \bigcap_n \bigcup_\nu\left[\left|X_{n+\nu}-X\right| \geqq \epsilon\right]
   & = \mu \bigcap_n \bigcup_\nu\left[\left|X_{n+\nu}\right| = 0 \right]                \\
   & = \mu \bigcap_n \bigcup_\nu \{A: A\subset \mathfrak a, A \ne \{1, \dots, n+\nu\}\} \\
   & = \mu \bigcap_n \Omega > 0.
\end{aligned}

If \(X = 0\), then

\begin{aligned}
  \mu \bigcap_n \bigcup_\nu\left[\left|X_{n+\nu}-X\right| \geqq \epsilon\right]
   & = \mu \bigcap_n \bigcup_\nu\left[\left|X_{n+\nu}\right| = 1 \right] \\
   & = \mu \bigcap_n \bigcup_\nu \{1, \dots, n+\nu\}                     \\
   & = \mu \bigcap_n \Omega > 0.
\end{aligned}

So failing to converge a.e. is obvious. \(\square\)

\begin{center}\rule{0.5\linewidth}{0.5pt}\end{center}

\hypertarget{exercise-12}{%
  \subsubsection{Exercise 12}\label{exercise-12}}

If \(X\) is integrable, then the set \([X \neq 0]\) is of
\(\sigma\)-finite measure. What if \(\int X\) exists?
(\(\mu[|X| \geqq c] \leqq \frac{1}{c} \int|X|\))

\textbf{Solution}

If \(X\) is integrable, then \(\int |X|\) is finite. Based on the hint,
\(\forall c \in \mathbb Z_+\),

\begin{aligned}
  \mu\left[|X| > c \right]
   & = \int \mathbb I_{\{|X| \ge c\}} d\mu                                                             \\
   & \le \frac{1}{c}\left[\int c \mathbb I_{\{|X| \ge c\}} d\mu + \int |X| I_{\{|X| < c\}} d\mu\right] \\
   & \le \frac{1}{c}\int |X| d\mu = \frac{1}{c}\int |X| < \infty.
\end{aligned}

Therefore, since

\[[X \ne 0] = \bigcup_{c=0}^\infty\left[ |X| > c \right],\]

we know that \([X \ne 0]\) is of \(\sigma\)-finite measure.

If only \(\int X\) exists, based on the definition, how about
\(\int_{-\infty}^\infty x  dx\) ? We cannot have \(\sigma\)-finiteness.
\(\square\)

\begin{center}\rule{0.5\linewidth}{0.5pt}\end{center}

\hypertarget{exercise-13-undone}{%
  \subsubsection{Exercise 13 (Undone)}\label{exercise-13-undone}}

Let \((T, \mathfrak T, \tau)\) be a measure space, to every point \(t\)
of which is assigned a measure \(\mu_t\) on \(\mathfrak a\). Let the
function on \(T\) defined by \(\mu_t A\) for any fixed \(A\) be
\(\mathfrak T\)-measurable.

The relation \(\mu A=\int_T \mu_t A d \tau(t)\) defines a measure
\(\mu\) on \(\mathfrak a\). If \(\int_{\Omega} X(\omega) d \mu(\omega)\)
exists, then the function defined on \(T\) by
\(U(t)=\int_{\Omega} X(\omega) d \mu_t(\omega)\) exists and is
\(\mathfrak T\)-measurable, and
\(\int_{\Omega} X(\omega) d \mu(\omega)=\int_T U(t) d \tau(t)\).

\begin{center}\rule{0.5\linewidth}{0.5pt}\end{center}

\hypertarget{exercise-14-2}{%
  \subsubsection{Exercise 14}\label{exercise-14-2}}

Let \(\varphi\) be the indefinite integral of \(X\). Express
\(\varphi^{+}, \varphi^{-}, \bar{\varphi}\) in terms of \(X\).

\textbf{Solution}

\[\varphi(\cdot) = \int_{\cdot} X d\varphi\]

Based on the definition of \(\varphi^+\), we know that for
\(A, B \in \mathfrak a\)

\[\varphi^+(A) = \sup_{B\subset A} \varphi(B) = \sup_{B\subset A} \int_{B} X d\varphi = \int_A X^+ d\varphi\]

Similarly, we know that

\[\varphi^-(\cdot) = \int_{\cdot} X^- d\varphi\]

And so

\[\bar\varphi(\cdot) = \int_{\cdot} (X^+ + X^-) d\varphi = \int_{\cdot} |X| d\varphi.\]

\(\square\)

\begin{center}\rule{0.5\linewidth}{0.5pt}\end{center}

\hypertarget{exercise-15}{%
  \subsubsection{Exercise 15}\label{exercise-15}}

If \(\int_A X_n \rightarrow 0\) uniformly in \(n\) as
\(\mu A \rightarrow 0\) or as \(A \downarrow \emptyset\), then the same
is true of \(\int_A\left|X_n\right| ;\) and conversely. Interpret in
terms of signed measures.

\[\left(\int_A\left|X_n\right|=\int_{A\left[X_n \geqq 0\right]} X_n-\int_{A\left[X_n<0\right]} X_n .\right)\]

\textbf{Solution}

This is closely related to the previous question. If
\(\varphi_n(A)\to 0\) uniformly, then

\begin{aligned}
           & \varphi^+(A) = \sup_{B\subset A} \varphi_n(A) \xrightarrow{\text u} 0,\quad \varphi^-(A) = -\inf_{B\subset A} \varphi_n(A)\xrightarrow{\text u} 0 \\
  \implies & \bar\varphi_n(A) = \varphi_n^+(A) + \varphi_n^-(A) \xrightarrow{\text u} 0                                                                        \\
  \implies & \int _A |X_n| \xrightarrow{\text u} 0.
\end{aligned}

The converse is obvious since \(|\int_A  X_n| \le \int_A |X_n| \).
\(\square\)

\begin{center}\rule{0.5\linewidth}{0.5pt}\end{center}

\hypertarget{exercise-16}{%
  \subsubsection{Exercise 16}\label{exercise-16}}

If finite \(\int_A X_n \rightarrow \int_A X\) finite, uniformly in
\(A(\in \mathfrak a)\), then
\(\int_{\Omega}\left|X_n-X\right| \rightarrow 0\) ; and conversely.

\textbf{Solution}

This is related to the proof of dominated convergence theorem. By the
given condition, uniformly in \(A\in \mathfrak a\),

\[\int_A (X_n - X) \xrightarrow{\text u } 0.\]

This gives us the right to choose \(A\) aribitrarily. And
\(\forall \epsilon > 0, A\in \mathfrak a\), \(\exists N\) such that when
\(n > N\),

\[\left|\int_A (X_n - X)\right| < \frac{\epsilon}{2}\]

Now for each \(n\), let \(A_{n_1} = [X_n \ge X], A_{n_2} = [X_n < X]\),
then as long as \(n > N\), we will have

\[\int_\Omega (X_n - X)^+ = \left|\int_{A_{n_1}} (X_n - X)\right|, \quad \int_\Omega (X_n - X)^- = \left|\int_{A_{n_1}} (X_n - X)\right|\]

And so

\[\int_\Omega |X_n - X| = \int_\Omega (X_n - X)^+ + \int_\Omega (X_n - X)^- < \epsilon\]

which means \(\int_{\Omega}\left|X_n-X\right| \rightarrow 0\).

The converse is true since \(\forall\) \(A \in \mathfrak a\)

\[\left|\int_A (X_n - X)\right| \le \int_A |X_n- X| \le \int_{\Omega} |X_n - X| \to 0\]

Therefore, \(\forall \epsilon > 0\), \(\exists N\) such that when
\(n > N\),

\[0 \le \left|\int_A (X_n - X)\right| \le \int_A |X_n- X| \le \int_{\Omega} |X_n - X| < \epsilon.\]

Since this applies to all \(A\)\textquotesingle s, we have uniform
convergence here.

\hypertarget{exercise-17}{%
  \subsubsection{Exercise 17}\label{exercise-17}}

If \(0 \leqq X_n \stackrel{\mu}{\rightarrow} X\), then finite
\(\int_{\Omega} X_n \rightarrow \int_{\Omega} X\) finite implies that
\(\int_A X_n \rightarrow\) \(\int_A X\) uniformly in \(A\) (also if
\(\stackrel{\mu}{\longrightarrow}\) is replaced by
\(\stackrel{\text { a.e. }}{\longrightarrow}\) )\\
\(\left(0 \leqq\left(X-X_n\right)^{+} \leqq X\right.\) integrable, and
\(\left.\int\left(X-X_n\right)^{+}-\int\left(X-X_n\right) \rightarrow 0.\right)\)

\textbf{Solution}

According to the given hint, we know that

\[(X - X_n)^+ \xrightarrow{\mu} 0, \quad 0 \le (X - X_n)^+ \le |X|,\]

where \(|X|\) is integrable. By the dominated convergence theorem, we
have that

\[\int\left(X-X_n\right)^{+}-\int\left(X-X_n\right) \rightarrow 0.\]

\end{document}
