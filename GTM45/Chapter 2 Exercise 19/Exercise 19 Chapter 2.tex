% Options for packages loaded elsewhere
\PassOptionsToPackage{unicode}{hyperref}
\PassOptionsToPackage{hyphens}{url}
%
\documentclass[
]{article}
\usepackage{amsmath,amssymb}
\usepackage{lmodern}
\usepackage{iftex}
\ifPDFTeX
  \usepackage[T1]{fontenc}
  \usepackage[utf8]{inputenc}
  \usepackage{textcomp} % provide euro and other symbols
\else % if luatex or xetex
  \usepackage{unicode-math}
  \defaultfontfeatures{Scale=MatchLowercase}
  \defaultfontfeatures[\rmfamily]{Ligatures=TeX,Scale=1}
\fi
% Use upquote if available, for straight quotes in verbatim environments
\IfFileExists{upquote.sty}{\usepackage{upquote}}{}
\IfFileExists{microtype.sty}{% use microtype if available
  \usepackage[]{microtype}
  \UseMicrotypeSet[protrusion]{basicmath} % disable protrusion for tt fonts
}{}
\makeatletter
\@ifundefined{KOMAClassName}{% if non-KOMA class
  \IfFileExists{parskip.sty}{%
    \usepackage{parskip}
  }{% else
    \setlength{\parindent}{0pt}
    \setlength{\parskip}{6pt plus 2pt minus 1pt}}
}{% if KOMA class
  \KOMAoptions{parskip=half}}
\makeatother
\usepackage{xcolor}
\setlength{\emergencystretch}{3em} % prevent overfull lines
\providecommand{\tightlist}{%
  \setlength{\itemsep}{0pt}\setlength{\parskip}{0pt}}
\setcounter{secnumdepth}{-\maxdimen} % remove section numbering
\ifLuaTeX
  \usepackage{selnolig}  % disable illegal ligatures
\fi
\IfFileExists{bookmark.sty}{\usepackage{bookmark}}{\usepackage{hyperref}}
\IfFileExists{xurl.sty}{\usepackage{xurl}}{} % add URL line breaks if available
\urlstyle{same} % disable monospaced font for URLs
\hypersetup{
  hidelinks,
  pdfcreator={LaTeX via pandoc}}

\author{}
\date{}

\begin{document}

\hypertarget{exercise-19}{%
  \subsubsection{\texorpdfstring{Exercise 19
    }{Exercise 19 }}\label{exercise-19}}

If the \(X_n\) are integrable and \(\lim \int_A X_n\) exists and is
finite for every \(A\), then the \(\int\left|X_n\right|\) are uniformly
bounded, \(\int_A\left|X_n\right| \rightarrow 0\) uniformly in \(n\) as
\(\mu A \rightarrow 0\) and as \(A \downarrow \emptyset\), and there
exists an integrable \(X\), determined up to an equivalence, such that
\(\int_A X_n \rightarrow \int_A X\) for every \(A\). (Use 18.)

\textbf{Solution}

Is it about uniformly integrability?

Using the conclusion of 14, we can define \(\varphi_n(A) = \int_A X_n\).
With the condition \(\mu A \to 0\) and \(A \downarrow \empty\), we have
that \(\varphi_n \to 0\). Conditioned on \(A\downarrow 0\),
\(\varphi_n A \to 0\) as \(\mu A \to 0\). Then, with the same logic for
exercise 18 of chapter 1, we know that \(\varphi_n A \to 0\) uniformly
as \(\mu A \to 0\) and \(A \downarrow \emptyset\) and
\(\varphi = \lim \varphi_n\) is \(\mu\)-continuous and
\(\sigma\)-additive.

The question here is why we need \(A\downarrow\emptyset\). Maybe because
we try to avoid things like Dirac function? But in the next question, it
seems that this condition could be suppressed when \(\mu\) is finite.

The fact that \(\varphi_n A \to 0\) uniformly as \(\mu A \to 0\) and
\(A \downarrow \emptyset\), plus the conclusion in exercise 15 implies
that \(\int_A\left|X_n\right| \rightarrow 0\) uniformly in \(n\) as
\(\mu A \rightarrow 0\) and as \(A \downarrow \emptyset\) (since
\(\varphi_n\)\textquotesingle s are all finite, then by exercise 9 of
chapter 1 it is true). The \(\mu\)-continuity and the
\(\sigma\)-additivity of \(\varphi\), along with the Radon-Nikodym
theorem tells us that \(\varphi\) is an indefinite integral of some
\(X\) up to an equivalence and

\[\varphi_n \to \varphi \implies \text{ for every A, } \int_A X_n \to \int_A X.\]

The only thing that remains to prove is that \(\int |X_n|\) are
uniformly bounded.

We can show that based on the given condition,
\(X_n \xrightarrow{\text{a.e.}} X\). If not, then we know that

\[\mu[X_n \not\to X] \ne 0\]

Therefore, we know that there is a set, say, \(A_0\) with
\(\mu A_0 = \delta_0 > 0\) such that \(X_n \not\to X\) on \(A_0\). This
means that there exists some \(\epsilon_0 > 0\), and for any \(N_0 > 0\)
there exists some \(n > N_0\) such that on \(A_0\)

\[|X_n - X| \ge \epsilon_0\]

If it is possible, then we could then find a infinite subsequence
\(X_{nj}, j = 1, 2, \dots\) such that \(|X_{nj} - X| \ge \epsilon_0\) on
\(A_0\) but \(\int_{B} X_{nj} \to \int_{B} X\) for all
\(B \subset A_0\). We need to argue then it is impossible.

Consider the subsequence of \(X_{nj}\), say \(X_{njk}\) where we have
\(\mu[X_{njk} - X \ge \epsilon_0] \ge \frac{\delta}{2}\). Without loss
of generality, assume this subsubsequence is infinite. Notice that we
still have \(\int_{B} X_{njk} \to \int_B X\) for all \(B\subset A_0\).
There must exists some \(B_{0}\subset A_0\) with
\(\mu B_0 \ge \frac{\delta}{2}\) such that on \(B_0\), there are
infinitely many \(X_{njk}\)\textquotesingle s such taht
\(X_{njk} - X \ge \epsilon_0\). Otherwise, there would exists a
\(B_1 \subset A_1\) such that there exists \(K_1\) such that when
\(k > K_1\), \(X_{njk} - X < \epsilon_0\) on \(B_1\). But this means
that for all \(k > K_1\), \(X_{njk} - X < \epsilon_0\) on \(A_0 - B_1\)
where \(\mu(A_0 - B_1) < \frac{\delta}{2}\) which is contradictory.

Now we obtain \(B_0\) with \(\mu B_0 \ge \frac{\delta}{2}\) and
\(X_{njk} - X \ge \epsilon_0\) on \(B_0\). This means that for any
\(K_2 > 0\), there exists some \(k > K_2\) such that,

\[\int_{B_0} (X_{njk} - X) d\mu \ge \frac{\delta\epsilon_0}{2}.\]

However, the given condition tells us that for
\(\frac{\delta\epsilon_0}{2} > 0\), there exists an \(K\) such that when
\(k > K\),

\[\int_{B_0} X_n - X < \frac{\delta \epsilon_0}{2}.\]

We have found a contradiction. This means that we cannot have
subsubsequence to be infinite. Then the subsubsequence for
\(\mu[X_{njk} - X < -\epsilon_0] \ge \frac{\delta}{2}\) cannot be
infinite for the same reason. Then the subsequence of \(X_{nj}\) cannot
be infinite. And so the whole assumption that \(X_n \not\to X\) on
\(A_0\) is invalid. In summary, we must have

\[X_n \xrightarrow{\text{a.e.}} X.\]

With a similar argument in exercise 17, we could use dominated
convergence theorem to control \((X - X_n)^+\) and \((X - X_n)^-\) by
\(|X|\) integrable to get

\[I_n := \int_{\Omega} |X_n - X| \to 0.\]

which means that there exists some \(N_1\) such that when \(n > N_1\),

\[I_n < \int |X|\]

Denote \(M = \max\{I_1, I_2, \dots, I_{N_1}, \int |X|\}\). Then

\[\int |X_n| \le \int |X_n - X| + \int |X| \le 2M,\]

which means that \(\int |X_n|\) uniformly bounded.

\end{document}
